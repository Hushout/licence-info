\chapter*{Introduction}
\addcontentsline{toc}{chapter}{Introduction}
\markboth{Introduction}{Introduction}
\label{chap:introduction}

Cette année, nous avons eu pour tâche de nous rassembler par groupe afin d'effectuer un projet TER tout au long du second semestre. Dans le cadre de ce dernier, nous avons eu l'opportunité d'avoir un enseignant et/ou chercheur nous encadrant. Celui-ci nous aidait alors chaque semaine lors de réunions sur certains points du projet, nous permettant, la plupart du temps, de nous débloquer et de nous aider à nous avancer sur la suite du projet en nous répartissant au mieux les tâches. \newline
Notre travail était donc porté sur un casse-tête appelé "Hashiwokakero" ou plus simplement "Hashi". Ainsi nous avons eu pour but de créer son résolveur afin de le résoudre le plus efficacement possible à l'aide de nos connaissances en algorithmique et en programmation C++. \newline
Concernant le déroulement du projet, nous avons choisi de concentrer notre programme sur GitHub. Ce choix a eu deux différentes raisons de naître, la première concernait l'utilisation de Git qui nous permettait alors de synchroniser le travail effectué à la fin de chaque session de programmation et la deuxième concernait le côté pratique puisque nous pouvons ainsi partager le travail entre nous, étudiants et encadrant, afin que ce dernier puisse voir notre avancement et que nous puissions voir celui de l'autre. \newline
De ce fait, tout cela nous a mené à créer notre projet et son rapport dont voici le plan. \newline
Tout d'abord, nous allons aborder les domaines de l'informatique dans lequel se situe notre projet en les présentant. Puis nous nous attarderons sur le problème précis sur lequel nous avons travaillé. Ensuite, nous assisterons à la description détaillée et expliquée de notre travail. Et pour finir, nous achèverons sur une conclusion faisant part de nos perspectives ouvertes par notre projet ainsi que l'avenir du programme. Nous pouvons aussi trouver à la toute fin de ce rapport, des remerciements ainsi que la bibliographie regroupant les ouvrages sur lesquels nous nous sommes appuyés.