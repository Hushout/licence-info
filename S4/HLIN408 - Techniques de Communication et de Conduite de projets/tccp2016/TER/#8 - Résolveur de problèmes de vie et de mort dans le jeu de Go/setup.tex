
%------------------------------------------------------------------------------
%       Inclusion des packages et parametrage
%------------------------------------------------------------------------------



    % -----------------     Génaral   ------------------ %
    % Police
    % http://tex.stackexchange.com/questions/44694/fontenc-vs-inputenc
    \usepackage[T1]{fontenc}
    \usepackage[utf8x]{inputenc}
    
    % Autorise les 'anciennes' commandes pour la table des matières.
    % Permet de basculer entre 'book' et 'scrbook'
    \usepackage{scrhack}
    
    
    % -----------------     Marges    ------------------ %
    % Permet de réduires les marges hautes et basses qui sont un poil trop grosses par défaut.
    \usepackage{geometry}
    \geometry{top=80pt, bottom=80pt}


    % -----------------     Aspect     ----------------- %
    % Modifie l'apect des 'captions' pour garder une cohérence avec le reste
    \usepackage[labelfont=bf, labelsep=colon, format=hang, textfont=singlespacing]{caption}
    
    % Modifie le comportement de reset de la numérotation des figures, des formules et des tableaux entre les chapitres.
    \usepackage{chngcntr}
    \counterwithout{equation}{chapter}
    \counterwithout{figure}{chapter}
    \counterwithout{table}{chapter}

    % Format global utilisé
    \setkomafont{chapter}{\normalfont\bfseries\huge}

    % Espace interlignes
    \usepackage{setspace}
    \onehalfspacing
    % \doublespacing  % Décommenter pour une version avec annotations (correction)

    % ------------------- functional, default-------------------
    \usepackage[dvipsnames]{xcolor}  % plus de couleurs
    \usepackage{array}  % format personnalisé sur les tableau (requis pour la table des matières).
    \usepackage{graphicx}
    \usepackage{subfig}
    \usepackage{amsmath}
    \usepackage{amsthm}
    \usepackage{amsfonts}
    \usepackage{calc}  % calculs anvant la compilation
    \usepackage[unicode=true,bookmarks=true,bookmarksnumbered=true,
                bookmarksopen=true,bookmarksopenlevel=1,breaklinks=false,
                pdfborder={0 0 0},backref=false,colorlinks=false]{hyperref}


%------------------------------------------------------------------------------
%       (re)new commands / settings
%------------------------------------------------------------------------------
    % ----------------- Référencement ----------------
    \newcommand{\secref}[1]{Section~\ref{#1}}
    \newcommand{\chapref}[1]{Chapitre~\ref{#1}}
    \renewcommand{\eqref}[1]{Équation~(\ref{#1})}
    \newcommand{\figref}[1]{Figure~\ref{#1}}
    \newcommand{\tabref}[1]{Tableau~\ref{#1}}

    % ------------------- Couleurs -------------------
    \definecolor{darkgreen}{rgb}{0.0, 0.5, 0.0}
    \definecolor{UniBlue}{RGB}{0, 74, 153}
    \definecolor{UniRed}{RGB}{193, 0, 42}
    \definecolor{UniGrey}{RGB}{154, 155, 156}


    % ------------------- Aspect -------------------
    % Évite les problèmes liés au placement éloigné des flottants.
    \let\mySection\section\renewcommand{\section}{\suppressfloats[t]\mySection}
    \let\mySubSection\subsection\renewcommand{\subsection}{\suppressfloats[t]\mySubSection}


    % ------ Commandes personnalisées (utiles pendant la création ------
    % Todo
    \newcommand{\todo}[1]{\textbf{\textcolor{red}{(TODO: #1)}}}
    \newcommand{\extend}[1]{\textbf{\textcolor{darkgreen}{(EXTEND: #1)}}}
    
    % Colore les botes
    \newcommand{\draft}[1]{\textbf{\textcolor{NavyBlue}{(DRAFT: #1)}}}



    % ------------------- pdf settings -------------------
    % Propriétées de PDF
    \hypersetup{pdftitle={Rapport TER},
                pdfauthor={Victor Huesca},
                pdfsubject={Rapport du TER du Jeu de Go},
                pdfkeywords={IA, Go, vie ou mort, algorithme, rapport},
                pdfpagelayout=OneColumn, pdfnewwindow=true, pdfstartview=XYZ, plainpages=false}


    % Pour la plupart des algo de ce rapport
    \usepackage[english, frenchb]{babel}
    \usepackage[french]{algorithme}


    % Pour les algo de compression (Le caption entre en conflie avec le paquet ci-dessus).
    \usepackage[french]{algorithm2e}
    \usepackage{algorithm}
    \usepackage{algorithmic}
    \usepackage[]{amssymb, amsmath}
    \usepackage{framed}
    \usepackage{natbib}

    % Francisation des algorithmes
    \renewcommand{\algorithmicrequire} {\textbf{\textsc{Entrées:}}}
    \renewcommand{\algorithmicensure}  {\textbf{\textsc{Sorties:}}}
    \renewcommand{\algorithmicwhile}   {\textbf{tantque}}
    \renewcommand{\algorithmicdo}      {\textbf{faire}}
    \renewcommand{\algorithmicendwhile}{\textbf{fin tantque}}
    \renewcommand{\algorithmicend}     {\textbf{fin}}
    \renewcommand{\algorithmicif}      {\textbf{si}}
    \renewcommand{\algorithmicendif}   {\textbf{finsi}}
    \renewcommand{\algorithmicelse}    {\textbf{sinon}}
    \renewcommand{\algorithmicthen}    {\textbf{alors}}
    \renewcommand{\algorithmicfor}     {\textbf{pour}}
    \renewcommand{\algorithmicforall}  {\textbf{pour tout}}
    \renewcommand{\algorithmicdo}      {\textbf{faire}}
    \renewcommand{\algorithmicendfor}  {\textbf{fin pour}}
    \renewcommand{\algorithmicloop}    {\textbf{boucler}}
    \renewcommand{\algorithmicendloop} {\textbf{fin boucle}}
    \renewcommand{\algorithmicrepeat}  {\textbf{répéter}}
    \renewcommand{\algorithmicuntil}   {\textbf{jusqu'à}}