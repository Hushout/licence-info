%% start of file `template.tex'.
%% Copyright 2006-2013 Xavier Danaux (xdanaux@gmail.com).
%
% This work may be distributed and/or modified under the
% conditions of the LaTeX Project Public License version 1.3c,
% available at http://www.latex-project.org/lppl/.


\documentclass[11pt,a4paper,sans]{moderncv}        % possible options include font size ('10pt', '11pt' and '12pt'), paper size ('a4paper', 'letterpaper', 'a5paper', 'legalpaper', 'executivepaper' and 'landscape') and font family ('sans' and 'roman')

% moderncv themes
\moderncvstyle{classic}                             % style options are 'casual' (default), 'classic', 'oldstyle' and 'banking'
\moderncvcolor{green}                               % color options 'blue' (default), 'orange', 'green', 'red', 'purple', 'grey' and 'black'
%\renewcommand{\familydefault}{\sfdefault}         % to set the default font; use '\sfdefault' for the default sans serif font, '\rmdefault' for the default roman one, or any tex font name
%\nopagenumbers{}                                  % uncomment to suppress automatic page numbering for CVs longer than one page

% character encoding
\usepackage[utf8]{inputenc}                       % if you are not using xelatex ou lualatex, replace by the encoding you are using
%\usepackage{CJKutf8}                              % if you need to use CJK to typeset your resume in Chinese, Japanese or Korean

% adjust the page margins
\usepackage[scale=0.75]{geometry}
%\setlength{\hintscolumnwidth}{3cm}                % if you want to change the width of the column with the dates
%\setlength{\makecvtitlenamewidth}{10cm}           % for the 'classic' style, if you want to force the width allocated to your name and avoid line breaks. be careful though, the length is normally calculated to avoid any overlap with your personal info; use this at your own typographical risks...

% personal data
\name{Wissem}{Soussi}
\title{Curriculum Vitæ}                               % optional, remove / comment the line if not wanted
\address{277 Avenue Paul Parguel}{34090 Montpellier}{France}% optional, remove / comment the line if not wanted; the "postcode city" and and "country" arguments can be omitted or provided empty
\phone[mobile]{+33~(0)~667~64~07~47}                   % optional, remove / comment the line if not wanted
%\phone[fixed]{+2~(345)~678~901}                    % optional, remove / comment the line if not wanted
%\phone[fax]{+3~(456)~789~012}                      % optional, remove / comment the line if not wanted
\email{wiwi.sam96@outlook.it}                               % optional, remove / 
% to show numerical labels in the bibliography (default is to show no labels); only useful if you make citations in your resume
%\makeatletter
%\renewcommand*{\bibliographyitemlabel}{\@biblabel{\arabic{enumiv}}}
%\makeatother
%\renewcommand*{\bibliographyitemlabel}{[\arabic{enumiv}]}% CONSIDER REPLACING THE ABOVE BY THIS

% bibliography with mutiple entries
%\usepackage{multibib}
%\newcites{book,misc}{{Books},{Others}}
%----------------------------------------------------------------------------------
%            content
%----------------------------------------------------------------------------------
\begin{document}
%\begin{CJK*}{UTF8}{gbsn}                          % to typeset your resume in Chinese using CJK
%-----       resume       ---------------------------------------------------------
\makecvtitle
\section{État Civil}
\cventry{nom}{Soussi}{}{}{}{}
\cventry{prénom}{Wissem}{}{}{}{}
\cventry{né le}{16/01/1996}{}{}{}{}
\cventry{nationalité}{Italienne}{}{}{}{}

\section{Éducation}
\cventry{Actuellement--2015}{Licence}{Université de Montpellier}{Montpellier (France)}{}{CMI (Cursus Master en Ingénierie) Informatique}  % arguments 3 to 6 can be left empty
\cventry{2015--2010}{Baccalauréat}{I.T.C.G Baggi}{Sassuolo (Italie)}{\textit{86/100}}{Systèmes informatiques d'entreprises}

\section{Expérience}
\subsection{Stage}
\cventry{25/08/2014--24/05/2014}{MARCA CORONA s.p.a}{Employer}{Sassuolo}{}{Stage de 9 semaines en Italie dans l'industrie de la céramique.\newline{}%
Objectifs:%
\begin{itemize}%
\item Apprendre toutes les démarches pour la création d'un carrelage, les différents pôles de l'industrie et les objectifs respectifs;
\item Stocker sur une base de données tous les documents relatifs à l'achat des matières premières et aux entretiens effectués sur les machines;
\item Création d'un outil logiciel (avec Microsoft Access) qui contient une base de données avec les informations des employés, les cours de sécurité professionnelle, les cours qu'ils ont suivis, leur date, lesquels n'ont pas encore été suivis et un aide pour la création d'un planning pour les cours futurs. 
\end{itemize}} 

\section{Langues}
\cvitemwithcomment{Italien}{Courant}{Langue maternelle}
\cvitemwithcomment{Arabe}{Courant}{Langue maternelle}
\cvitemwithcomment{Français}{DELF B2 (diplôme français)}{Niveau universitaire}
\cvitemwithcomment{Anglais}{Bonne compréhension et production oral et écrite}{Niveau universitaire}



\section{References}
\begin{cvcolumns}
  \cvcolumn{Tuteurs de stage}{\begin{itemize}\item Eugenia Marchi\end{itemize}}
  \cvcolumn{Professeurs universitaires}{entre autres:\begin{itemize}\item Anne-Elisabeth Baert\item Pierre Pompidor\end{itemize}(autres noms si demandé)}
\end{cvcolumns}

% Publications from a BibTeX file without multibib
%  for numerical labels: \renewcommand{\bibliographyitemlabel}{\@biblabel{\arabic{enumiv}}}% CONSIDER MERGING WITH PREAMBLE PART
%  to redefine the heading string ("Publications"): \renewcommand{\refname}{Articles}
\nocite{*}
\bibliographystyle{plain}
\bibliography{publications}                        % 'publications' is the name of a BibTeX file

% Publications from a BibTeX file using the multibib package
%\section{Publications}
%\nocitebook{book1,book2}
%\bibliographystylebook{plain}
%\bibliographybook{publications}                   % 'publications' is the name of a BibTeX file
%\nocitemisc{misc1,misc2,misc3}
%\bibliographystylemisc{plain}
%\bibliographymisc{publications}                   % 'publications' is the name of a BibTeX file

\end{document}


%% end of file `template.tex'.
