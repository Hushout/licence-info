\documentclass[tikz]{standalone}

\usepackage[utf8]{inputenc}
\usepackage[T1]{fontenc}
\usepackage{lmodern}

\usepackage[french]{babel}
\usetikzlibrary{positioning}
\usetikzlibrary{shapes}
\usetikzlibrary{calc}

\tikzset{
    >=latex,
    event/.style={
        circle,
        inner sep=2pt
    },
    period/.style={
        rounded rectangle,
        minimum height=4pt,
        inner sep=0pt
    },
    queue elt/.style={
        draw,
        minimum width=.6cm,
        minimum height=.6cm
    }
}

\begin{document}
\begin{tikzpicture}

\newcommand{\emptyqueue}[1]{
    \path[draw, thick]
        (#1) --++ (1.4, 0)
        ++ (-1.4, .6) --++ (1.4, 0)
        ++ (-.5, -.1) --++ (-.4, -.4)
        ++ (.4, 0) --++ (-.4, .4);
};

\newcommand{\queue}[3]{
    \path[draw, thick]
        coordinate (#1) at ($(#2)+(.4,0)$)
        (#2) --++ (#3, 0)
        ++ (-#3, .6) --++ (#3, 0);
};

% Flèches du temps
\path[draw, thick]
    coordinate (main-task) at (0, 0)
    (0, 0) --++ (12, 0)
    coordinate (main-task-mid);

\path[draw, dashed, ->, thick]
    (main-task-mid) --++ (2, 0) node[yshift=.2cm, left]
    {\footnotesize Tâche principale};

\path[draw, thick]
    coordinate (send-task) at (0, -3)
    (0, -3) --++ (12, 0)
    coordinate (send-task-mid);

\path[draw, dashed, ->, thick]
    (send-task-mid) --++ (2, 0)
    node[yshift=-.4cm, left, align=right, font=\footnotesize]
    {Tâche d'envoi\\et de réception};

% Réception de la requête B
\node[
    event, fill=red!60!black,
    label={[above=0cm] \small Requête B}
] (req-B) at (.5, 0) {};

% File d'attente initiale
\queue{queue-1}{1, -1.8}{2}
\node[queue elt, anchor=south west] (queue-1B) at (queue-1) {B};
\node[queue elt, anchor=south west] (queue-1A) at ($(queue-1)+(.6,0)$) {A};

\node[font=\footnotesize, yshift=.6cm] at ($(queue-1A)!.5!(queue-1B)$)
    {File d'attente};

\draw[->, rounded corners=4pt] (req-B) |- (queue-1B.west);

% Traitement de la requête A
\node[
    period, fill=red!60!black, minimum width=2cm, xshift=.75cm,
    label={[below=.3cm] \small Traitement de A}, anchor=west
] (traitement-A) at (queue-1A.east|-send-task) {};

\draw[->, rounded corners=4pt] (queue-1A.east) -| (traitement-A.west);

\node[
    event, fill=red!60!black,
    label={[above=0cm] \small Réponse A}
] (res-A) at (traitement-A.east|-main-task) {};

% Réception de la requête C
\node[
    event, fill=blue!60!black,
    label={[above=.4cm, name=req-C-label] \small Requête C}
] (req-C) at (traitement-A|-main-task) {};

\draw[dotted] (req-C-label) -- (req-C);

% File d'attente numéro 2
\queue{queue-2}{$(queue-1-|traitement-A.east)+(.5,0)$}{2}
\node[queue elt, anchor=south west] (queue-2C) at (queue-2) {C};
\node[queue elt, anchor=south west] (queue-2B) at ($(queue-2)+(.6,0)$) {B};

\draw[->, rounded corners=4pt] (req-C) |- (queue-2C.west);

\draw[->] (traitement-A.east) -- (res-A)
    node [above=.1cm, midway, sloped, fill=white] {notifie};

% Traitement de la requête B
\node[
    period, fill=blue!60!black, minimum width=2cm, xshift=.75cm,
    label={[below=.3cm] \small Traitement de B}, anchor=west
] (traitement-B) at (queue-2B.east|-send-task) {};

\draw[->, rounded corners=4pt] (queue-2B.east) -| (traitement-B.west);

\node[
    event, fill=blue!60!black,
    label={[above=0cm] \small Réponse B}
] (res-B) at (traitement-B.east|-main-task) {};

\draw[->] (traitement-B.east) -- (res-B)
    node [above=.1cm, midway, sloped] {notifie};

% File d'attente numéro 3
\queue{queue-3}{$(queue-2-|traitement-B.east)+(1.1,0)$}{1.4}
\node[queue elt, anchor=south west] (queue-3C) at (queue-3) {C};

\end{tikzpicture}
\end{document}
