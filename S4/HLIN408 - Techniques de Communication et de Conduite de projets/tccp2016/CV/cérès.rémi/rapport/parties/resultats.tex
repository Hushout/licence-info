\chapter{Bilan et difficultés rencontrées}

\section{Bilan de l'avancement du projet}

Nous avons terminé l'implémentation du module de communication via HTTP. Pour tester ce module, nous avons développé un programme de test effectuant des échanges avec le serveur \url{www.example.com} sur différents types de requêtes. Nous nous sommes ainsi assurés du bon fonctionnement de ce module, y compris lorsque plusieurs requêtes sont envoyées en parallèle. Désormais, le module est utilisable pour l'envoi de requêtes avec n'importe quel verbe sur n'importe quelle URL.

Nous avons également fini la création du module des services, et créé une classe permettant d'utiliser le service de stockage Dropbox. Nous avons écrit un programme de test qui récupère le quota, liste un répertoire, envoie des fichiers et en supprime sur un compte Dropbox. Ce test permet également de valider le bon fonctionnement du module HTTP qui est utilisé par la classe.

À l'heure où nous écrivons ces lignes, Arcus est capable de communiquer avec Dropbox. Nous travaillons actuellement sur l'intégration de OneDrive, le service de Microsoft, que nous comptons terminer pour le rendu du livrable. Ce travail permettra de montrer la facilité avec laquelle un nouveau service peut être intégré dans l'interface multi-services, et ce sans aucune modification dans les autres modules.

Nous avons achevé la réalisation du module de journalisation des actions de l'application, et celui-ci est utilisé à travers les autres modules. Nous avons également développé un programme pour tester la création du journal et la cohérence des messages affichés.

Nous avons aussi terminé la mise en place du module permettant de manipuler les configurations locales et distantes, y compris l'intégration avec la bibliothèque de lecture et d'écriture du JSON et SQLite~3. Nous avons vérifié que le comportement de ce module soit correct avec un programme de tests écrivant et lisant des configurations locales et créant des bases de données.

Enfin, au moment de la rédaction de ce rapport, nous sommes en train de finaliser le moteur de synchronisation afin de faire fonctionner notre application. Nous prévoyons de finaliser ce dernier module dans les semaines à venir.

Sur les modules que nous avons mis en place et testés, nous nous sommes efforcés de régler tous les bogues que nous avons pu rencontrer, et nous n'avons pas pour le moment connaissance des éventuels bogues restants.

\section{Difficultés rencontrées}

La conception de notre application a été une première difficulté. En effet, nous n'avions encore jamais travaillé sur la conception d'une application de cette taille. Nous avons donc passé plusieurs semaines à définir le cahier des charges et les modèles afin d'éviter de nous bloquer par la suite en poursuivant avec un modèle erroné. Nous les avons ensuite raffinés pendant la phase d'implémentation.

Au cours de la création du module HTTP de notre application, nous avons été confrontés pour la première fois à la programmation concurrente. Nous avons donc dû apprendre les concepts liés à ce type de programmation, tels que les tâches, les méthodes de synchronisation de ces tâches, les promesses et valeurs futures. Nous avons rapidement rencontré les problèmes couramment associés aux tâches et à leur synchronisation, tels que les \emph{deadlocks} ou les \emph{race conditions}. Ce nouvel apprentissage a constitué pour nous une difficulté importante, et nous sommes parvenus à la surmonter grâce à l'aide de notre encadrante.
