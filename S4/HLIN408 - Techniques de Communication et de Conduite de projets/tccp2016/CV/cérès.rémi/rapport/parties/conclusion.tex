\chapter*{Conclusion et perspectives}
\addcontentsline{toc}{chapter}{Conclusion et perspectives}

\section*{Connaissances et apprentissages}

Ce projet nous a permis de mettre en pratique et de consolider de nombreuses connaissances aquises tout au long de notre parcours universitaire. Nous donnons dans cette partie des exemples d'enseignements qui nous ont été utiles dans la réalisation de ce projet.

L'enseignement en systèmes d'information et bases de données~(HLIN304) nous a appris a concevoir une base de données, à la gérer et à l'interroger avec le langage SQL. Cela nous a permis de concevoir la base de données de notre application.

Nous avons appris à programmer en C++ à l'aide des enseignements de programmation impérative~(HLIN202, HLIN302). Nous avons pu utiliser la programmation orientée objet et modéliser nos modules en UML grâce aux cours de modélisation et de programmation par objet~(HLIN406).

Les cours de techniques de communication, de conduite de projets~(HLIN408) et de projet C.M.I.~(HLSE205) nous ont fait découvrir la gestion de projet et permis d'apprendre à utiliser de nombreux outils utiles tels que \LaTeX, Git, GitLab, Make et gdb...

Enfin, nous avons découvert et appris à nous servir du format JSON avec l'enseignement «~du binaire au web~»~(HLIN102).

Ce projet nous a également permis de développer de nouvelles compétences. En particulier, tout ce qui concerne le réseau a été une découverte pour nous. Nous avons appris les principes de HTTP, exploré l'A.P.I. Rest fournie par Dropbox et OneDrive et étudié le fonctionnement de la bibliothèque cURL. Nous avons également appris les bases de la programmation concurrente.

Enfin, ce projet a été l'occasion de découvrir le fonctionnement et l'interface de SQLite~3. Il nous a permis de voir une première utilisation concrète des bases de données.

\section*{Perspectives}

La seconde partie de notre projet, que nous réaliserons en stage au laboratoire d'informatique, de robotique et de microélectronique de Montpellier (Lirmm) entre le 30~mai et le 30~juin, sera l'occasion d'implémenter les fonctionnalités encore manquantes à notre application.

Nous nous occuperons de la sécurité de l'application, notamment du chiffrage des informations d'authentification et des données de l'utilisateur. Nous doterons également notre application d'une interface graphique simple permettant de faciliter sa configuration. Nous prévoyons également de porter notre application sur plusieurs systèmes. Enfin, nous prévoyons d'intégrer notre application avec plus de services.
