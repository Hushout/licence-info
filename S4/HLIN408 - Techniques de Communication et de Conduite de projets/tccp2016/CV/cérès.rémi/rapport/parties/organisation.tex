\chapter{Organisation du projet}

\section{Méthode et organisation du travail}

Lors du développement d'Arcus, nous avons décidé de travailler un maximum de temps ensemble et de manière très régulière. Pour que nous puissions profiter tous deux des nombreuses connaissances que nous apporte le développement de ce projet, nous avons favorisé la méthode de programmation en binôme (ou \emph{pair programming}). Cette méthode consiste à donner la responsabilité de l'écriture du code à l'un des deux développeurs et à faire relire le code produit au fur et à mesure par le second développeur. Une étude menée par l'IEEE Computer Society~\parencite{mcdowell2003} a en effet montré que les étudiants utilisant cette méthode produisaient du code de meilleure qualité.

Afin d'être le plus efficace et d'avancer le plus rapidement possible nous nous sommes réunis quotidiennement. Durant les jours de la semaine, nous nous sommes vus entre 16~h~30 et 19~h~30, afin de faire le point sur l'avancement du projet, de définir les objectifs du jour et de les réaliser. Enfin, chaque week-end, nous avons réalisé les tâches en attente que nous n'avions pas pu faire durant la semaine.

Toutes les deux semaines, nous nous sommes réunis avec notre encadrante Mme~Hinde~Bouziane afin de faire le point sur l'état d'avancement de l'application. Cette réunion bihebdomadaire nous a également permis de bénéficier des ses conseils et de son aide sur les difficultés que nous avons rencontrées lors du développement.

\section{Répartition du travail dans le temps}

Nous avons découpé le développement du projet Arcus en deux périodes de temps. La première s'inscrit dans le cadre du T.E.R., et s'est déroulée de janvier à mi-mai 2017. La seconde aura lieu pendant un stage au Laboratoire d'Informatique, de Robotique et de Microélectronique de Montpellier (Lirmm) encadré par Mme~Hinde~Bouziane de fin-mai à fin-juin 2017. Par ailleurs, nous avons déjà effectué une semaine de stage au laboratoire au cours de la première semaine des vacances de printemps.

\clearpage

% Nous avons défini des objectifs pour chacune de ces différentes périodes.
Le but de la première période est d'obtenir une première version fonctionnelle de l'application qui répond à une partie du cahier des charges : la mise en place d'une interface commune aux différents services de stockage, la répartition des données de l'utilisateur sur ces services, leur réplication et le fonctionnement en tâche de fond.
% Au cours de la seconde période, nous finaliserons l'application en apportant les fonctionnalités manquantes : le chiffrement des données, le portage de l'application sur d'autres systèmes d'exploitation et une interface graphique pour l'application.

Ce rapport rend compte de notre avancement au cours de la première partie, c'est-à-dire pendant le projet T.E.R. Nous avons découpé cette période de travail en plusieurs phases.

\begin{enumerate}[label=\textbf{\arabic*.}]
    \item \textbf{Préparation du projet.} Nous avons écrit le cahier des charges de l'application, choisi les outils de travail et les principales technologies utilisées. Nous avons fait une première version du diagramme de répartition des tâches dans le temps, et une première modélisation de l'architecture de l'application.
    \item \textbf{Développement du projet.} Nous avons implanté les fonctionnalités de l'application en raffinant la modélisation au fur et à mesure. Pour chaque module implanté, nous nous sommes efforcés d'écrire des tests afin de s'assurer de leur bon fonctionnement.
    \item \textbf{Finalisation du projet.} Cette phase a consisté en la correction de bogues afin d'obtenir une version suffisamment stable pour pouvoir être présentée en vue de la soutenance et du rendu du projet T.E.R.
\end{enumerate}

L'annexe ~\ref{appendix:gantt} détaille cette répartition sous forme d'un diagramme de Gantt.

\section{Outils de travail collaboratif}

Nous avons choisi d'utiliser Git\footnote{Git : \url{https://git-scm.com/}} au travers du serveur GitLab\footnote{GitLab : \url{https://gitlab.info-ufr.univ-montp2.fr/matteodelabre/arcus/}} hébergé par le Service Informatique de la Faculté (Sif). Le logiciel Git, libre, permet la gestion des versions du projet et facilite la collaboration entre nous, notamment lorsque nous travaillions en même temps sur deux machines différentes. Les serveurs GitLab sont quant à eux basés sur un logiciel libre également et le service est fourni gratuitement par le Sif.

Pour communiquer entre nous à distance, nous avons utilisé Telegram\footnote{Telegram : \url{https://telegram.org/}} pour les messages écrits et appear.in\footnote{appear.in : \url{https://appear.in/}} pour la communication orale et le partage d'écrans.

Pour prendre des notes pendant les réunions, nous avons utilisé le langage Markdown\footnote{Markdown : \url{https://daringfireball.net/projects/markdown/}} qui permet un formattage rapide. Pour rédiger les différents documents, y compris ce rapport, nous avons utilisé \LaTeX\footnote{\LaTeX : \url{https://www.latex-project.org/}} pour sa capacité à produire des documents de bonne qualité.

Enfin, pour éditer le code du projet et les documents, nous nous sommes servis de l'éditeur de texte Atom\footnote{Atom : \url{https://atom.io/}}, logiciel libre développé par GitHub. Il s'agit en effet de l'éditeur que nous utilisons habituellement.
