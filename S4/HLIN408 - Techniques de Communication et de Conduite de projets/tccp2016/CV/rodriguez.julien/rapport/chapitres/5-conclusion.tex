\chapter{Conclusion}\label{chap:conclusion}
    \section{Bilan}
        \paragraph{}Nous sommes heureux d'annoncer que le Tsumego fonctionne pour le problème du 6 dans le coins, il détermine donc si le groupe va survivre ou pas. Du début jusqu'à la fin de ce projet les membres sont restés actifs ce qui nous à permis d'arriver à faire fonctionner notre algorithme de solution sur certains problèmes de Tsumego. C'est une réussite personnelle pour chaque membre car ce sujet de TER nous paraissait insurmontable en vu des échecs des précédentes années. Mais ce projet nous a intéressé et au delà du défi de réussir il nous a permis de nous enrichir autant techniquement que socialement. Au fur et à mesure du temps nous avons appris à nous connaître et à nous adapter avec le caractère des uns, parfois solitaire, et avec celui des autres. Et c'est ce que nous noterons de plus important lors d'un travail de groupe c'est de pouvoir s'adapter pour maintenir la cohésion.
            
    \section{Perspectives}
        \paragraph{}Nous aurions pu améliorer la performance du Tsumego en éliminant les gobans des fils. Ceci permettrait une économie de mémoire lors de l'appel récursif du tsumego. En effet nous n'aurions pas à recalculer l'intégralité des fils mais seulement ceux qui nous intéressent. 
        
        \paragraph{}De plus nous avions la possibilité d'utiliser le résultat du Tsumego pour créer une intelligence artificielle. Celle-ci aurait dû être capable de jouer contre une personne sur des problèmes de petite taille. Nous disposons de deux possibilités pour mettre en place cette IA. La première serait de lancer le Tsumego une seule fois puis d'utiliser l'arbre développé pour répondre aux coups du joueur. La seconde, plus économe en mémoire, serait l'utilisation d'une version alternative du Tsumego gardant en mémoire seulement le coup gagnant. Elle lancerait alors le Tsumego  après chaque coup du joueur.
        
        \paragraph{}Un tout autre cas de figure est aussi envisageable : ajouter à notre application une fonctionnalité de jeu en ligne. Cette possibilité permettrait de nous initier à la communication internet d'une application et serait un prolongement intéressant pédagogiquement parlant. De plus en poursuivant ce projet nous pourrions envisager de le publier en ligne et permettre à des amateurs de Go de jouer sur un logiciel libre et gratuit !

    \section{Apport personnels du projet}
        \paragraph{}Effectuer ce projet fut, pour chacun de nous, un gain en connaissances énorme. Il nous a permis d'en apprendre plus dans de nombreux domaines ainsi que de se perfectionner dans ceux que nous connaissions déjà. Cet apport ne s'est pas cantonné à l'informatique mais il a touché d'autre niveaux tels que la mise en page en Latex, dans le graphisme, la musique pour certains, la rédaction de rapport, etc. \\
        Au final les projets TER sont un apport considérable dans nos études. Ils nous mettent face à la réalité du travail en groupe et nous apporte de nombreuses compétences. C'est l'un des points fort de notre formation et cela fait même parfois émerger des vocations pour certains.
