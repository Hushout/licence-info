%!TEX TS-program = xelatex
\documentclass[]{friggeri-cv}
\addbibresource{bibliography.bib}
\usepackage[utf8]{inputenc} 
\usepackage[french]{babel}

\begin{document}
\header{Julien}{Rodriguez}
       {Etudiant Informatique}


% In the aside, each new line forces a line break
\begin{aside}
  \section{}
    12 rue Roland Garros
    34470 Pérols
    France
    ~
    \href{mailto:julien.ro34@gmail.com}{julien.ro34@gmail.com}
 %   \href{http://}{}
  %  \href{http://}{}
  \section{Langues}
    Français
    Anglais E: B1
    Espagnol : B1
  \section{Programation}
    {\color{red} $\varheartsuit$} Java
    Python, C, C++
    Scheme, \LaTeX, MySQL 
    CSS3 \& HTML5
\end{aside}

\section{interêt}

{\textbf{Judo - Jiu Jitsu} : Je pratique le Judo depuis l'âge de 7 ans. }

{\textbf{Piano} : J'ai joué au Rockstore à Montpellier en 2013. Je suis actuellement dans un groupe nommé "The Outdoors" dont je suis le fondateur.}

{\textbf{Course à pied} : Je cours depuis l'age de 12 ans à raison d'une fois par semaine. }

\section{education}

\begin{entrylist}
  \entry
    {2011--2014}
    {Baccalauréat avec mention.}
    {Lycée Jean François Champollion, Lattes}
    {\emph{Série scientifique option Science de l'ingénieur.}}
  \entry
    {2015–2017}
    {Série Informatique}
    {Faculté des Sciences de Montpellier.}
    {}
  \end{entrylist}

\section{experience}

\begin{entrylist}
  \entry
    {2010}
    {Aéroport Montpellier Méditéranée, Mauguio}
    {Stagiaire}
    {\emph{Suivis de différents secteurs.}}
  \entry
    {2014}
    {Auchan, Pérols}
    {Saisonier}
    {\emph{Mise en rayon au BSA (Boissons sans Alcool).}}
  \entry
    {2015}
    {ENFP Group, Montpellier.}
    {Stagiaire}
    {\emph{Développement de cours en support pour de l'e-learning.}}
\end{entrylist}

\section{Divers}


\begin{entrylist}
  \entry
    {2014}
    {Projet}
    {Montpellier}
    {Développement du jeu "Tic Tac Toe" en C++ avec la bibliothèque SFML (Simple and Fast Multimedia Library).}
  \entry
    {2015}
    {Projet}
    {Montpellier}
    {Développement du jeu "FullSquare" en Java. Ceci est un projet personnel disponible sur le PlayStore.}
  \entry
    {2016}
    {Projet}
    {Montpellier}
    {Développement du jeu "1010" en C++ avec la bibliothèque ncurses.}
\end{entrylist}

%\section{publications}

%Put your publications here!

%%% This piece of code has been commented by Karol Kozioł due to biblatex errors. 
% 
%\printbibsection{article}{article in peer-reviewed journal}
%\begin{refsection}
%  \nocite{*}
%  \printbibliography[sorting=chronological, type=inproceedings, title={international peer-reviewed conferences/proceedings}, notkeyword={france}, heading=subbibliography]
%\end{refsection}
%\begin{refsection}
%  \nocite{*}
%  \printbibliography[sorting=chronological, type=inproceedings, title={local peer-reviewed conferences/proceedings}, keyword={france}, heading=subbibliography]
%\end{refsection}
%\printbibsection{misc}{other publications}
%\printbibsection{report}{research reports}

\end{document}
