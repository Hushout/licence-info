\documentclass[11pt,origdate]{lettre}
\usepackage{palatino}
\usepackage[T1]{fontenc}
\usepackage[utf8]{inputenc}
\usepackage[frenchb]{babel}
\usepackage{graphicx}

\begin{document}
	\begin{letter}{Monsieur Stéphane Bessy, \\
	Monsieur William Puech,\\
	Département Informatique de la Faculté des Sciences \\
    Université de Montpellier -- Bt. 16 - CC 12 \\
    Place Eugène Bataillon \\
    34095 Montpellier cedex 05} 
		\address{Julien Lesinski \\
		30 rue de la pinède, \\
		résidence La Pinède, bâtiment A\\
		13270 Fos-Sur-Mer, France}
		\name{Julien Lesinski} % Mon nom
		%\location{Mon d�partement dans l'entreprise}
		\telephone{(+33)6 09 54 06 90}
		\email{lesinski.julien@hotmail.com}
		\nofax % pas de fax. Alternative: \fax{numero}
		
		\francais % Met les labels en fran�ais et le \closing{} en pleine largeur
		          % Variantes: \anglais, \americain, et \allemand
		\pagestyle{empty} % alternatives : plain (num�ro de page en pied), headings (ent�te avec lieu et date)
		
		\conc{}
		\lieu{Montpellier}
		\date{\today}
		\signature{\includegraphics[height=1cm]{signature}}
	
		% MISE EN PAGE ET PERSONNALISATION DU PACKAGE
		
		\renewcommand{\concname}{} % Elimine l'affichage du texte "Objet :"
		\renewcommand{\emaillabelname}{} % Elimine l'affichage du texte "E-mail :"
		
		%\def\openingspace{10mm} % ajuste l'espace vertical autour du champ sujet, default: 1cm
 		\def\sigspace{10mm} % Espacement vertical entre texte et signature(s), default: 1.5cm
		
		\makeatletter
		% BOITE D'ENTETE
		\def\pict@let@width{185}       % default: 185
		\def\pict@let@height{65}       % default: 65
		\def\pict@let@hoffset{0}       % default: 0
		\def\pict@let@voffset{0}       % default: 0
		% TRAIT DE PLIAGE
		%\def\rule@hpos{-25}            % default: -25
		%\def\rule@vpos{-15}            % default: -15
		%\def\rule@length{10}           % default: 10
		% ADRESSE DE L'EXPEDITEUR
		\def\fromaddress@let@hpos{-10} % default: -10
		\def\fromaddress@let@vpos{80}  % default: 70
		                               % je remonte légèrement l'adresse vers le coin supérieur gauche
		\fromaddress@let@width=69mm    % default: 69
		% LIEU D'EXPEDITION
		\def\fromlieu@let@hpos{90}     % default: 90
		\def\fromlieu@let@vpos{0}      % default: 62 
		                               % je déplace lieu et date sous l'adresse du destinataire
		\fromlieu@let@width=69mm       % default: 69
		% ADRESSE DU DESTINATAIRE
		\def\toaddress@let@hpos{90}    % default: 90
		\def\toaddress@let@vpos{40}    % default: 40
		\toaddress@let@width=80mm      % default: 80
		\makeatother


		% DEBUT DE LA LETTRE
		\opening{Messieurs,}
		
		\par Je souhaite continuer mon cursus en informatique à l'Université de Montpellier et plus particulièrement au travers du Master Informatique parcours IMAGINA que vous proposez car c'est une des seules formations en France (dans le secteur publique) qui permet de former aux métiers du jeu vidéo ainsi qu'à la conception 3D.

	    \par J'ai effectué ma première année de licence informatique à Marseille. Pour ensuite faire les deux autres à Montpellier car la ville est plus adaptée aux étudiants mais surtout par rapport au fait que le Master Informatique IMAGINA ne se fait qu'à Montpellier.

		\par C'est grâce à ma passion pour l'informatique (en particulier les possibilités que permettent la programmation), et, bien sûr grâce aux compétences acquises lors de mon parcours en Licence Informatique que je me permet de postuler pour ce Master. Vous pouvez compter sur mes connaissances en programmation et graphisme 2D, ma créativité, mon sérieux et ma motivation pour réussir cette formation.
		
		\closing{Je vous prie d'agréer, Messieurs, l'expression de ma considération distinguée.}

	\end{letter}
\end{document}