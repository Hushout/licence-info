\part{Formalisation du langage bio-logique}
\label{formalisme}

\section{Définitions}
\label{sec:formalisme/definitions}

\subsection{Alphabet}
\label{subsec:formalisme/definitions/alphabet}
\par
Soit $S$ l'ensemble des \textbf{Symboles} formant, avec le \textbf{mot vide}, l'\textbf{Alphabet}. Ils sont qualifiés de \textbf{Forward} ou \textbf{Reverse} : c'est leur sens.
\par
Les différentes couleurs de site représentent les différentes \textbf{Inputs}, au nombre de $n$.

\begin{wrapfigure}[10]{R}{8cm}
\centering
\begin{tabular}{R{2.4cm} C{1.2cm} C{1.2cm} C{2.1cm}}
\toprule
& \textbf{Forward} & \textbf{Reverse} & \textbf{Taille (kb\protect\footnotemark)}\\
\midrule
\rowcolor{GT1} \textbf{Promoteurs} & \PF & \PR & \textit{0,04}\\
\rowcolor{GT2} \textbf{Terminateurs} & \TF & \TR & \textit{0,10}\\
\rowcolor{GT1} \textbf{Gènes} & \GF & \GR & \textit{1,00}\\
\rowcolor{GT2} \textbf{Sites} & \SF{3} & \SR{3} & \textit{0,04}\\
\rowcolor{GT1} \textbf{Sites utilisés} & \UF{3} & \UR{3} & \textit{0,04}\\
\bottomrule
\end{tabular}
\caption{\label{fig:formalisme/definitions/alphabet}Alphabet du langage}
\end{wrapfigure}
\footnotetext{kb : kilobases ; en milliers de bases}

\par
Un \textbf{Mot du langage} est le résultat de la \textbf{concaténation} de plusieurs symboles ou mots du langage.
\par
On parlera de \textbf{Séquence} pour désigner des mots composés uniquement de promoteurs, terminateurs et gènes.

\subsection{Mots bien formés}
\label{subsec:formalisme/definitions/mots_bien_formes}
\par
On associe à chaque symbole un attribut de \textbf{Taille}, et on définit la \textbf{distance entre deux symboles} par la somme des tailles des symboles entre les deux.\\
\par
On attribue aux promoteurs une information de \textbf{Portée}, qui nous permet de définir la notion de \textbf{croisement des promoteurs} : deux promoteurs se croisent s'ils sont de sens opposés en vis-à-vis (\PF \ANY \PR) et séparés d'une distance inférieur à la portée d'un promoteur uniquement par le mot vide ou des sites. On fixe cette portée à \textbf{1,5 kilobases}.
\par
Un mot est dit \textbf{bien formé} si pour chaque input, il contient exactement deux sites, et s'il ne contient aucun croisement de promoteurs.\\
\exx{\PF \PR \GF ~n'est pas un mot bien formé car il y a croisement de promoteurs.}
\exx{\PF \TF \PR \GF ~est un mot bien formé.}

\section{Propriétés des mots bien formés}
\label{sec:formalisme/proprietes}

\subsection{Excision et Inversion}
\label{subsec:formalisme/proprietes/excision}
\par
Un mot bien formé de $n$ inputs possède $2*n$ sites. Ces inputs peuvent être ou non \textbf{activées}, ce qui applique une transformation aux symboles situés entre les deux sites de l'input, et qui transforme aussi ces sites en sites utilisés.\\

\begin{itemize}
\item \SF{3} \ANY \SF{3} ~et \SR{3} \ANY \SR{3} ~appliquent une \textbf{Excision} à \ANY ~si l'input est activée, c'est-à-dire que l'on supprime le sous-mot \ANY.\\
\exx{\SF{0} \PF \GF \SF{0} ~devient \UF{0} \UF{0}}
\item \SF{3} \ANY \SR{3} ~et \SR{3} \ANY \SF{3} ~appliquent une \textbf{Inversion} à \ANY ~si l'input est activée, c'est-à-dire que l'on effectue une rotation du sous-mot \ANY~, puis que l'on change le sens de tous ses symboles.\\
\exx{\SF{0} \PF \GF \SR{0} ~devient \UF{0} \GR \PR \UR{0}}
\end{itemize}

\subsection{Non-croisement des sites}
\label{subsec:formalisme/proprietes/croisement}
\par
Un mot bien formé respecte le \textbf{non-croisement des sites}, c'est-à-dire que les paires de sites se terminent dans l'ordre dans lequel elles ont commencé.\\
\exx{\SF{0} \SF{1} \SR{0} \SR{1} ~n'est pas un mot bien formé car il y a croisement des sites.}
\exx{\SF{0} \SF{1} \SF{2} \SF{2} \SF{1} \SF{0} ~et \SF{0} \SF{1} \SF{1} \SF{2} \SF{2} \SF{0} ~sont des mots bien formés.}
\par
En assimilant chaque paire de site à une paire de parenthèse, on peut ainsi dire que l'ensemble des mots bien formés composés uniquement de sites forme un \textbf{langage de Dyck}\protect\footnotemark.
\footnotetext{Le langage de Dyck est l'ensemble des mots bien parenthésés, sur un alphabet composé des parenthèses ouvrante et fermante}

\subsection{Confluence des activations}
\label{subsec:formalisme/proprietes/confluence}
\par
Dans un mot bien formé, \textbf{l'ordre d'activation} des inputs ne change pas le mot obtenu.\\
\exx{\SF{0} \PF \SF{1} \TF \SF{1} \SR{0} ~peut devenir \SF{0} \PF \UF{1} \UF{1} \SR{0} ~ou \UF{0} \SR{1} \TR \SR{1} \PR \UF{0} ~qui deviennent tous deux \UF{0} \UF{1} \UF{1} \PR \UR{0}}

\vspace{0.2cm}
\subsubsection*{\textit{Dans la suite du document et sauf mention contraire, les mots sont considérés comme bien formés.}}