\chapter*{Introduction}
\addcontentsline{toc}{chapter}{Introduction}
\markboth{Introduction}{Introduction}
\label{chap:introduction}
%Introduction

Nous avons travaillé sur la programmation d'un jeu Pong, un précurseur de l'industrie du jeu vidéo telle que nous la connaissons. Ce jeu est une référence à tel point que désormais tout joueur le considère comme un classique du jeu. Notre problème est que justement il fallait rendre ce jeu devenu culte plus moderne et accessible sans pour autant le dénaturer. Nous avons donc décidé de le remettre au goût du jour en le rendant plus intuitif, en implémentant une interface graphique différente de celle que nous avons l'habitude de voir, plus dynamique, mais surtout jouable à plusieurs en utilisant le modèle des jeux ".io". Le déroulement du projet , s'est fait via la méthode agile. Nous avons donc tout de suite mis en évidence différentes étapes majeures à réaliser pour la réussite de ce projet. Nous avons donc principalement travaillé sur le mode multijoueur, la physique du rebond, l'algorithme de prévision inclus dans l'intelligence artificielle, les graphismes et les différents modes de jeu.\\
Le projet a donc évolué autour de ces étapes et grâce aux réunions hebdomadaires la cohésion du groupe était plus importante ce qui eut pour effet d'améliorer notre productivité et d'enrichir notre inspiration pour le projet.\\
Ce rapport portera sur deux grands axes qui sont la gestion du projet, décrivant notre méthode travail et nos outils de travail, et l'élaboration de notre projet où nous détaillerons toutes les étapes de la programmation du PongIO.