\part*{Conclusion}
\label{conclusion}

\markright{}
\addcontentsline{toc}{part}{Conclusion}

\section{État et ouverture du projet}
\label{subsec:conclusion/ouverture}
\par
Une fois notre programme capable de représenter des mots et les méthodes pour les manipuler, il a été aisé de faire la première partie du projet, la conversion d'un mot en une fonction logique. \textbf{La partie centrale et principale du projet a alors été de mettre en place cette génération et de la rendre viable à exécuter.} Une fois fait, à l'aide de la BDD, le passage d'une fonction logique à des mots s'est résumé à l'exécution de simples requêtes.
\par
A mesure des différentes versions de l'algorithme, nous avons réduit le temps de génération, mais celui-ci reste très élevé ; il s'agit donc de le diminuer. On pourrait déjà chercher un moyen d'éliminer les symétriques avant l'étape de génération, mais aussi améliorer la parallélisation des tâches, ou encore déterminer des profils de séquences éliminatoires. La génération au-delà de 3 inputs sera alors envisageable.
\par
Une autre ouverture possible serait, à partir des résultats de la génération, d'extrapoler des règles permettant de construire directement les meilleurs mots pour une fonction logique.

\section{Retours d'expérience}
\label{subsec:conclusion/retours}
\par
Ce projet a suscité chez nous trois un grand enthousiasme, et nous avons apprécié travailler en équipe.
\par
Le choix de ce sujet a été motivé par l'envie de répondre à un réel besoin, manifesté par un suivi en continu de notre avancée, et par une évolution des attentes exprimées lors de deux réunions avec l'intéressée.
\par
L'utilisation d'un large panel d'outils techniques nous a également permis d'approfondir nos connaissances sur le C++ et son univers. Se confronter aux limites de la puissance d'un ordinateur nous a aussi poussé à avoir une réflexion sur l'optimisation des algorithmes.