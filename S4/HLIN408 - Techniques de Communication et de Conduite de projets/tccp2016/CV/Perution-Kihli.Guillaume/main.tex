%% start of file `template.tex'.
%% Copyright 2006-2013 Xavier Danaux (xdanaux@gmail.com).
%
% This work may be distributed and/or modified under the
% conditions of the LaTeX Project Public License version 1.3c,
% available at http://www.latex-project.org/lppl/.


\documentclass[11pt,a4paper,sans]{moderncv}        % possible options include font size ('10pt', '11pt' and '12pt'), paper size ('a4paper', 'letterpaper', 'a5paper', 'legalpaper', 'executivepaper' and 'landscape') and font family ('sans' and 'roman')

% moderncv themes
\moderncvstyle{banking}                            % style options are 'casual' (default), 'classic', 'oldstyle' and 'banking'
\moderncvcolor{blue}                                % color options 'blue' (default), 'orange', 'green', 'red', 'purple', 'grey' and 'black'
%\renewcommand{\familydefault}{\sfdefault}         % to set the default font; use '\sfdefault' for the default sans serif font, '\rmdefault' for the default roman one, or any tex font name
%\nopagenumbers{}                                  % uncomment to suppress automatic page numbering for CVs longer than one page

% character encoding
\usepackage[utf8]{inputenc}                       % if you are not using xelatex ou lualatex, replace by the encoding you are using
%\usepackage{CJKutf8}                              % if you need to use CJK to typeset your resume in Chinese, Japanese or Korean

% adjust the page margins
\usepackage[scale=0.75]{geometry}
%\setlength{\hintscolumnwidth}{3cm}                % if you want to change the width of the column with the dates
%\setlength{\makecvtitlenamewidth}{10cm}           % for the 'classic' style, if you want to force the width allocated to your name and avoid line breaks. be careful though, the length is normally calculated to avoid any overlap with your personal info; use this at your own typographical risks...

% personal data
\name{Guillaume}{Pérution-Kihli}
%\title{Curriculum vitae}                               % optional, remove / comment the line if not wanted
\address{4bis rue Baudin}{34000 Montpellier}{France}% optional, remove / comment the line if not wanted; the "postcode city" and and "country" arguments can be omitted or provided empty
\phone[mobile]{06 99 82 27 55}                    % optional, remove / comment the line if not wanted
\email{guillaume.pk@gmail.com}                               % optional, remove / comment the line if not wanted             % optional, remove / comment the line if not wanted
\photo[64pt][0.4pt]{picture}                       % optional, remove / comment the line if not wanted; '64pt' is the height the picture must be resized to, 0.4pt is the thickness of the frame around it (put it to 0pt for no frame) and 'picture' is the name of the picture file
%\quote{Some quote}                                 % optional, remove / comment the line if not wanted

% to show numerical labels in the bibliography (default is to show no labels); only useful if you make citations in your resume
%\makeatletter
%\renewcommand*{\bibliographyitemlabel}{\@biblabel{\arabic{enumiv}}}
%\makeatother
%\renewcommand*{\bibliographyitemlabel}{[\arabic{enumiv}]}% CONSIDER REPLACING THE ABOVE BY THIS

% bibliography with mutiple entries
%\usepackage{multibib}
%\newcites{book,misc}{{Books},{Others}}
%----------------------------------------------------------------------------------
%            content
%----------------------------------------------------------------------------------
\begin{document}
%\begin{CJK*}{UTF8}{gbsn}                          % to typeset your resume in Chinese using CJK
%-----       resume       ---------------------------------------------------------
\makecvtitle

\section{Études / Diplômes}
\cventry{2015--2017}{Université de Montpellier}{Licence 2 Informatique}{Montpellier}{}{}
\cventry{2015}{Université de Montpellier}{Master 1 de Droit de l'entreprise et des affaires}{Montpellier}{}{}
\cventry{2014}{Université de Montpellier}{Licence de Droit privé}{Montpellier}{}{}
\cventry{2010}{Lycée Nevers}{Baccalauréat économique et social}{Montpellier}{\textit{Mention Bien}}{}

\section{Expériences}
\cventry{2017}{Université de Montpellier}{Projet de Licence 2}{Montpellier}{}
{\begin{itemize}
\item {Formalisation de la création de systèmes logiques sur un code génétique}
\item {Implémentation en C++11}
\end{itemize}}

\cventry{2014--aujourd'hui}{}{Auto-entrepreneur}{Montpellier}{}
{\begin{itemize}
\item {Création de sites web}
\end{itemize}}

\cventry{2008--aujourd'hui}{}{Webmaster de Parrain-Linux.com}{Montpellier}{}
{\begin{itemize}
\item {Création du site en 2008 (programmation depuis 0)}
\item {Gestion du site et mises à jour régulières}
\end{itemize}}

\section{Langues}
\cvitemwithcomment{Anglais}{Lu, écrit, parlé}{}

\section{Compétences informatiques}
\cvdoubleitem{Systèmes}{Linux, Windows}{Retouche d'images}{Gimp}
\cvdoubleitem{Langages}{C++11, PHP 5, JS, HTML, CSS}{Bureautique}{Libre Office, Microsoft Office}

\section{Occupations}
\cvitem{Programmation}{}
\cvitem{Lecture}{Actualité informatique, économie, etc}
\cvitem{Webmaster}{www.parrain-linux.com}
\cvitem{Associations}{Secrétaire du think-tank Responsable, Secrétaire du Mouvement des étudiants européens et démocrates}

% Publications from a BibTeX file without multibib
%  for numerical labels: \renewcommand{\bibliographyitemlabel}{\@biblabel{\arabic{enumiv}}}% CONSIDER MERGING WITH PREAMBLE PART
%  to redefine the heading string ("Publications"): \renewcommand{\refname}{Articles}
\nocite{*}
\bibliographystyle{plain}
\bibliography{publications}                        % 'publications' is the name of a BibTeX file

% Publications from a BibTeX file using the multibib package
%\section{Publications}
%\nocitebook{book1,book2}
%\bibliographystylebook{plain}
%\bibliographybook{publications}                   % 'publications' is the name of a BibTeX file
%\nocitemisc{misc1,misc2,misc3}
%\bibliographystylemisc{plain}
%\bibliographymisc{publications}                   % 'publications' is the name of a BibTeX file

\clearpage
%-----       letter       ---------------------------------------------------------
% recipient data
\recipient{Département Informatique}{Bâtimenet 16, faculté de sciences de Montpellier}
\date{9 février 2017}
\opening{Madame, Monsieur,}
\closing{Je vous prie d'agréer, Madame, Monsieur, mes salutations distinguées.}
\enclosure[Ci-joint]{curriculum vit\ae{}}          % use an optional argument to use a string other than "Enclosure", or redefine \enclname
\makelettertitle


je me permets de vous écrire afin de vous proposer ma candidature en tant qu’étudiant au sein de
votre Master.

Étudiant en informatique à la faculté de sciences de Montpellier, très passionné par ce domaine, je serais honoré de me voir intégrer votre établissement afin de poursuivre ma formation.

Ce serait pour moi l’occasion d’approfondir des matières qui, de part et d’autre m’intéressent et maintiennent ma curiosité.

En espérant pouvoir vous présenter de vive voix ce projet et mon entoushiasme envers ce Master lors d’un entretien, je vous joins également mon curriculum vitæ.


\makeletterclosing

%\clearpage\end{CJK*}                              % if you are typesetting your resume in Chinese using CJK; the \clearpage is required for fancyhdr to work correctly with CJK, though it kills the page numbering by making \lastpage undefined
\end{document}


%% end of file `template.tex'.
