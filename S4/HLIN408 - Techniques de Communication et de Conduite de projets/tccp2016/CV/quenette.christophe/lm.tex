\documentclass[12pt]{lettre}
\usepackage[utf8]{inputenc}
\usepackage[T1]{fontenc}
\usepackage[francais]{babel}

\begin{document}
\begin{letter}{Monsieur Pierre POMPIDOR \\ Co-responsable du master informatique \\ Faculté des sciences \\ Université de Montpellier \\
Place Eugène Bataillon \\ 34090 Montpellier}
   \date{Jeudi 9 février 2017}                       
   \nolieu{}                       
   \name{Christophe QUENETTE}
   \address{75 Avenue Augustin Fliche \\ 34090 Montpellier}
   \notelephone{} 
   \nofax{}             
   \email{christophe.quenette@gmail.fr}             
   \opening{ Objet : candidature au Master Aigle \\ À l'intention de Monsieur Pompidor, co-responsable du master }
   
   \begin{small}
   Monsieur,
   Je suis à présent en troisième année de Licence informatique à l’Université de Montpellier. Ainsi je vous adresse mon dossier de candidature pour le Master AIGLE que la faculté des sciences et le départemtent informatique propose, sous votre responsabilité.

Comme vous le savez, au cours de mes trois années de licence, j’ai eu l’occasion de bénéficier d’un cursus fondé sur des domaines variés de l'informatique.

Particulièrement intéressé par le Master AIGLE , j’ai profité de mes années universitaires à la faculté des sciences pour approfondir mes connaissances, notamment dans les domaines de la programmation du Web, des résaux, de la programation impérative...

Effectuer le Master AIGLE serait ainsi un complément essentiel à mes connaissances acquises et les enseignements dispensés s’inscriraient parfaitement dans la logique de mon projet professionnel. En effet une voie qui m'intéresse est travailler dans le développement du Web et des technologies qui s'y rapporte. Je suis aussi attiré a par tout ce qui s'apparente à l'administration et à la sécurité des résaux.

Disposant de la capacité de travail et de la motivation nécessaires à la réussite d’un master, je sollicite votre bienveillance à l’égard de mon dossier pour mon admission au Master AIGLE.
   \end{small}
  
   \closing{Je vous prie d’agréer, Monsieur, l’expression de ma considération respectueuse.}
\end{letter}
\end{document}