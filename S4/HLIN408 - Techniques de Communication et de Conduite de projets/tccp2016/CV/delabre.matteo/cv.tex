\documentclass[sans]{moderncv}

% Packages
\usepackage[a4paper,margin=2cm]{geometry}
\usepackage[utf8]{inputenc}
\usepackage[T1]{fontenc}
\usepackage{lmodern}
\usepackage[french]{babel}

% Configuration
\moderncvstyle{classic}
\moderncvcolor{burgundy}
\settowidth{\hintscolumnwidth}{0000 --- 0000}

\newcommand{\fillmid}{\hfill--\hfill}
\newcommand{\stackdate}[2]%
    {\parbox[t]{9mm}{\centering{#1\par\vspace{-1mm} \small{#2}}}}

% Données du CV
\name{Mattéo}{Delabre}
\title{Étudiant en informatique}
\address{2B impasse de la Mouta}{34880 Lavérune, France}{}
\phone{+33 7 81 68 59 52}
\email{contact@matteodelabre.me}
\social[linkedin]{matteodelabre}
\social[github]{matteodelabre}

\begin{document}

% Section de titre
\makecvtitle

% Liste des emplois
\section{Expérience professionnelle}

\cventry%
{\stackdate{2016}{mai}\fillmid{}prés.}%
{Administrateur systèmes et réseaux}%
{\newline{}Service informatique du lycée Jules Guesde}%
{Montpellier}{}%
{Embauché à mi-temps à la suite d’un stage de six semaines.
\begin{itemize}
    \item Administration des différents sous-réseaux informatiques du lycée.
    \item Maintenance et renouvellement des 650 ordinateurs (système et matériel).
    \item Assistance aux 2\,500~utilisateurs dans l'utilisation des nouvelles technologies.
    \item Gestion de la sécurité du réseau.
\end{itemize}}

% Liste des formations
\section{Formation}

\cventry%
{\stackdate{2015}{sept.}\fillmid{}prés.}%
{Cursus Master en Ingénierie (CMI) informatique}%
{\newline{}Université de Montpellier}%
{Montpellier}{}%
{Formation d’ingénieur en informatique appuyée sur la licence et les masters d’informatique de la faculté des sciences.
\begin{itemize}
    \item Enseignements d’informatique~: algorithmique et complexité, modélisation objet, bases de données, programmation en C++, Scheme, Java et Python.
    \item Enseignements d’ingénierie~: gestion de projet, communication, management, comptabilité.
    \item Projets annuels en groupe, au total 450 heures.
    \item Présentation de la formation lors des journées portes ouvertes.
\end{itemize}}

\cventry%
{\stackdate{2012}{sept.}\fillmid\stackdate{2015}{juillet}}%
{Baccalauréat scientifique, mention très bien}%
{\newline{}Lycée Jules Guesde}%
{Montpellier}{}{}

% Projets informatique
\section{Projets}

\cventry%
{\stackdate{2017}{janvier}\fillmid{}prés.}%
{Arcus -- Synchronisation multi-services de stockage dans les nuages}%
{}{}{}%
{Développement d’une application cliente permettant de synchroniser fichiers et répertoires avec plusieurs services de stockage dans les nuages. Favorise l’indépendance de l’utilisateur sur les fournisseurs de services en distribuant les données et en les chiffrant.}

\cventry%
{\stackdate{2016}{mars}\fillmid\stackdate{2016}{mai}}%
{Skizzle -- Jeu de plateformes coopératif}%
{}{}{}%
{Création d’un jeu inspiré des jeux de plateformes et faisant collaborer deux joueurs. Équipe de trois étudiants, développement en C++ sur une durée de 8~semaines. Conception d’un moteur physique et utilisation de la SFML pour les graphismes.\\\url{https://github.com/matteodelabre/projet-cmi/}}

% Langues parlées
\section{Langues}

\cvitem{Français}{Langue maternelle.}
\cvitem{Anglais}{Expérimenté en lecture et écriture de textes techniques. Compétent à l’oral.}

\section{Compétences informatiques}

\cvitem{Web}{Création de sites avec HTML, CSS et JavaScript. Bonne connaissance de WordPress. Gestion de serveurs Apache et Node.JS.}
\cvitem{\LaTeX{}}{Rédaction de documents tels que ce CV et de présentations avec Beamer.}
\cvitem{Systèmes}{Mise en place et maintenance de serveurs Windows~Server~2012 et Debian~8. Utilisation personnelle d’Ubuntu.}

\end{document}
