\documentclass{article}
\usepackage[francais]{babel}
\usepackage[utf8]{inputenc}
\usepackage[T1]{fontenc}
\usepackage{pdfpages}	
\usepackage{titlesec}
\usepackage{sectsty}
\usepackage{xcolor}
\usepackage{pst-all,pst-eucl}
\usepackage{hyperref}
\usepackage{ragged2e}
\usepackage{fancyhdr}
\usepackage{fullpage}
\usepackage{url}

\begin{document}
\section{Introduction}
Bonjour, présentations.
À l'heure actuelle, internet fait parti de notre quotidien en France mais cela n'est pas le cas pour tout le monde. En effet, certaines zones n'ont pas d'accès à certaines technologies et plus particulièrement n'ont pas accès à internet.
Lors de navigation sur internet, notre vie privée peut être exposée et donc atteinte de la même façon que notre propriété intellectuelle.
C'est pourquoi dès les années 80, la Commission nationale de l'informatique et des libertés (CNIL) nacquît pour mettre en place le droit numérique. 
Le droit numérique regroupe 3 points principaux qui sont la protection de la vie privée, protection de la propriété intellectuelle, et l’accessibilité numérique contre la fracture numérique qui sera le plan que nous allons suivre.
\section{Protection Vie privée}
\subsection{Vie privée}
\begin{itemize}
    \item L'origine de vie privée: Ivème siècle avant Jésus Christ avec Aristote qui différencie la vie de la sphère sociale publique (polis qui signifie cité)  et le concept de vie/sphère privée et domestique (oikos qui signifie maison).
    \item Définition: C’est un droit civil pour le respect des activités qui concerne l’intimité d’un individu. 
\end{itemize}
\subsection{Droit de la vie privée}
\subsubsection{Droit international}
Selon la déclaration universelle des droits de l’homme (C’est une déclaration de droits créée par
l’ONU) de 1948 :
"Nul ne sera l'objet d'immixtions arbitraires dans sa vie privée, sa famille, son domicile ou sa correspondance, ni d'atteintes à son honneur et à sa réputation. Toute personne a droit à la protection de la loi contre de telles immixtions ou de telles atteintes." 
Ceci implique que notre vie est protégée au niveau international.
\subsubsection{Droit Européen}
C'est ce qu'on appelle Droit au respect de la vie privée: Article 9 de la convention européenne de la sauvegarde des droits de l'Homme 
" Appartient à la vie privée tout ce qui n'a pas à être dans la sphère publique" 
Malheureusement en Europe on n'a pas de Définition exacte mais les tribunaux vont aller traiter au cas par cas.
\subsubsection{Droit Français}
l'article 6 du code civil :
Le respect au droit de la vie privée en droit Français: "Tout ce qui nécessitait pas sans l'accord de son auteur et qui n'est pas utile au domaine public"
Par exemple: L'homosexualité de Filipo révélée par closer, c'était pas utile dans la sphère publique (on avait pas besoin de savoir car
c'était un homme politique)
Et donc ça c'était une atteinte à la vie privée
Contre-exemple : Dans une manifestation, si on ne souhaite pas y être mais que des journalistes prennent une photo d'ensemble et qu'on s'y retrouve dessus. 
Repression pénale: Article 226-1
" Est puni d'un an d'emprisonnement et de 45~000 euros d'amende le fait au moyen d'un procédé quelconque, volontairement de porter atteinte à l'intimité de la vie privée d'autruit:
1/ En captant, enregistrant ou transmettant, sans le consentement de leur auteur, des paroles prononcées à titre privé ou confidentiel;
2/ En fixant, enregistrant ou transmettant, sans le consentement de celle-ci, l'image d'une personne se trouvant dans un lieu privé.\\
Lorsque les actes mentionnés au présent article ont été accompli au vu et au su des intéressés sans qu'ils s'y soient opposés, alors qu'ils étaient en mesure de le faire, le consentement de ceux-ci est présumé.
Ce n'est pas une atteinte à la vie privée car on s'est affiché soi même dans un lieu publique donc ce n'est pas les journalistes qui ont portés atteintes à notre vie privée 
Attention, bien faire attention à la mince frontière entre liberté d'expression et atteinte à la vie privée.
Chacun est libre de s'exprimer librement, de dire ses pensées, ses opinions etc. C'est un principe constitutionnel (c'est dans la charte constitutionnelle française).  Néanmoins, la liberté d'expression ne peut pas porter atteinte à la vie privée. 
\subsection{Protection de la vie privée en informatique}
Des mentions légales sont mises en place pour perpétuer le droit de la vie privée.
Le principe est de maîtriser le type d'informations collectées et par qui. Le but est d'aussi maîtriser ce qui est fait de nos informations.
Par exemple, l'utilisateur peut éviter qu'un directeur bloque l'accès pour un futur emploi, ou encore il peut éviter d'être interpelé par téléphone mais aussi sur internet ou dans la rue. On peut aussi consulter les informations nous concernant et les rectifier.
Sur internet, on peut partager tout ce qu'on souhaite. Cependant, parfois, la vie privée est atteinte. 
(manque d'équilibre entre secu+ protection de la vie privée \& partage sur internet)
Les utilisateurs d'internet donnent eux même leurs données  comme par exemple sur les réseaux sociaux. 
Lorsque l'on adhère à un certaines fonctionnalités d'internet (comme par exemple FaceBook), on accepte des contrats et alors, les données sont collectées à notre insu par des services.

Mais il y a eu de grands progrès dans le droit de vie privée en numérique.
Tout internaute a le droit de supprimer ses données gratuitement depuis une loi passée en 2010.
Cette loi "du droit à l'oubli numérique" a pour objectif de protéger la vie privée sur internet et de faciliter la suppression des données personnelles publiées sur internet.
Le CNIL a joué un rôle important dans cette loi et souhaite intégrer la loi du droit à l'oubli numérique dans la constitution.
Pour le moment ce projet reste en stand by car de nombreux sites dont mondiaux et donc les lois ne sont pas le même dans le monde qu'en Europe ou encore en France. 
De nombreuses entreprises ne peuvent désormais collecter des informations personnelles de façon restreintes et seulement si la personne fait
\begin{itemize}
    \item l’achat de biens et de services
    \item l’enregistrement pour des produits ou pour un compte
    \item l’accès à des services personnalisés
    \item l’enregistrement de garantie
    \item l’enregistrement pour des concours 
    \item une réclamation
    \item pour une assistance technique ou contacter le Service clientèle
    \item utiliser les forums
\end{itemize}
Exemple pris du site Blizzard Entertainment.
Cependant, les entreprises se doivent de mettre à disposition des "mentions légales" qui stipulent ce que nous acceptons (nous acceptons de donner certaines données stipulées par exemple adresse) et ces mentions légales se doivent de fournir ce que l'entreprise en faire.
Par exemple, dans l'entreprise \textit{Blizzard Entertainment}, celle-ci stipule que ".Blizzard pourra utiliser ces données pour générer des statistiques sur sa communauté de joueurs et les fournir à ses partenaires commerciaux. De plus, Blizzard pourra également utiliser ces informations pour renforcer la sécurité, l’intégrité du système (prévention du piratage, de la triche, etc.) ou à des fins de renforcement."
Blizzard nous renseigne aussi sur "Qui rassemble, traite et utilise vos informations ?"
Celui-ci répond par "  vous ne partagez ces informations qu’avec Blizzard, sauf indication contraire explicite. Certains services sont cependant fournis en collaboration avec des sociétés partenaires"
"Les données que vous envoyez par nos formulaires sur les sites de Blizzard ne seront envoyées qu’à Blizzard après avoir cliqué sur « Envoyer » ou « OK »"
"Souvenez-vous que les annonceurs et les sites Internet rattachés à nos sites par un lien peuvent rassembler des données personnelles sur vous. Nous vous rappelons que la Politique de confidentialité de Blizzard ne s’applique pas à ces annonceurs et à ces autres sites puisque nous ne possédons sur eux aucun contrôle."
Enfin bref, un long texte doit être à la disposition de l'utilisateur sur tout ce qui l'attend en acceptant les conditions et mentions légales.
\subsection{Dangers et atteintes de la vie privée}
Malheureusement il existe des atteintes à la vie privées auxquels on peut extraire 2 noms populaires comme PRISM et USA PATRIOT Act.
PRISM est un programme de surveillance américain qui collecte les renseignements et informations personnelles des individus sur internet.
PRISM a été au coeur d'une grande polémique... 
USA PATRIOT Act est une loi antiterroriste américaine qui donne aux services de sécurité le droit d'accès aux données détenues par les entreprises et particuliers sans autorisation judiciaire préalable et sans avertir les utilisateurs.
Cela pose des problèmes à un grand nombre d'entreprises ayant leurs données dans un hébergeur américain et cela même si leurs filiales sont à l'étranger. Par conséquent, l'administration américaine a accès à leurs données.
Malheureusement, on n'est pas à l'abri en étant en France car on utilise beaucoup de services hébergés aux Etats-Unis. et d'ailleurs, la majorité des services que l'on utilise sont Étasunien. 

\section{Protection propriété intellectuelle}
\subsection{Propriété intellectuelle}
\subsubsection{Ce que c'est etc}
Ce terme désigne toutes œuvres de l’esprit, c’est-à-dire inventions, œuvres littéraires et artistique,
dessins et modèles, noms et images données dans le commerce (par exemple logo).
Selon l'INPI (Institut national de la propriété industrielle), La propriété intellectuelle regroupe la propriété industrielle et la propriété littéraire et artistique. La propriété industrielle a plus spécifiquement pour objet la protection et la valorisation des inventions, des innovations et des créations.
Cette propriété intellectuelle est protégée par la loi avec l’aide des brevets, des droits d’auteurs, d’enregistrement de marques. Ceux-ci permettent aux créateurs de tirer une reconnaissance (qui peut-être financière).
On peut déposer nos projets sur le site de l'INPI afin d'obtenir des droits dessus.
La propriété intellectuelle sert à protéger les créations intellectuelles. Elle récompense l’effort des innovateurs en leur donnant des droits, leur permettant de diffuser leurs créations dans la société en les faisant fructifier, grâce à un monopole d'exploitation pour une période déterminée.
L'INPI mène une lutte contre la "contrefaçon": coopération internationale, renforcement de la législation nationale ou sensibilisation du grand public
OMPI : organisation mondiale de la propriété intellectuelle créée en 1967
Celle-ci sert à recenser toutes les créations intellectuelles et ce au niveau mondial (ou du moins dans
les 171 pays ayant signé la convention de Bern).
Si des individus ne respectent pas la loi, donc par exemple s’ils font du plagiat, ils devront déverser une amende jusqu’à 300K euros et 3 ans de prison

\subsubsection{Le libre}
Les logiciels libres sont devenus populaires à partir des années 1984 avec le projet GNU.
Le but était de permettre à tout le monde d'utiliser les logiciels librement mais aussi d'apporter les modifications que les gens souhaitent.
Libre garantie 3 libertés: Liberté d'utilisation (On peut utiliser le logiciel comme on le souhaite, même à des fins commerciales)
Liberté de modification (on peut modifier le logiciel à notre guise)
Liberté de distribution (On peut distribuer le logiciel en totalité ou en partie)
Attention à ne pas confondre avec un logiciel open-source:
opensource: un logiciel dont le code source est ouvert pour permettre à plusieurs personnes de travailler dessus.
La licence libre est une licence permettant de donner des droits d'auteur à une propriété intellectuelle.
Cette licence nécessite d'être au moins composée de 4 droits fondamentaux: 
0- usage de l'oeuvre
1- Etude de l'oeubre pour en comprendre le fonctionnement ou l'adapter à ses besoins
2- Modification de l'oeuvre (amélioration, extension, transformation)
3- Redistribution de l'oeuvre, même de façon commerciale
Si une oeuvre donne le droit d'utiliser, d'étudier, de modifier et de diffuser son oeuvre, 
lorsqu'une personne décide de le redistribuer celle-ci est obligée de marquée la nouvelle oeuvre par un copyLeft
On parle de copyleft fort lorsque les redistributions de l'oeuvre ne peuvent se faire que sous la licence initiale (même modifiée)
On parle de copyLeft faible ou standard lorsque les redictributions de l'oeuvre peuvent être ajoutés sous d'autres licences voire sous des licences propriétaires.

\subsection{Hadopi et le téléchargement illégal}
téléchargement illégal = acquérir des oeuvres protégées par des propiétés intelelctuelles (droits d'auteur) sans rémunérer de quelqconque façon les auteurs et producteurs

En 2013 : on estime 13 millions de Fr/mois sur un site de téléchargement illégal (données médiamétrie)

Rq : les gens disposent d'un droit à la copie privée qui consiste à pouvoir copier à des fins privées des oeuvres acquises légalement

Mise en place de la hadopi (haute autorité pour la diffusion des oeuvres et la protection des droits sur internet) poru lutter face à cela. Fondée en 2009, elle scanne les flux d'adresses IP sur les site de partage de fichiers et recherche lees adresses qu'elle trouve souvent. 

Si tu t'es fait chopper :
- 1 mail est envoyé (rappel à l'ordre)
- si dans les 6 mois suivants ils te trouent toujours : 2 ème mail
- si dans les 12 mois suivants le premier mail, il y a un 3ème rappel, alors il peut y avoir des poursuites pénales. Tribunal de police -> jusqu'a 1500 euros d'amende.

à l'origine, pouvait couper la co internet, finalement annulé (décret européen)

Aujourd'hui vivement contestée (mais également depuis son origine). Coupes budgétaires (11,4 M en 2012, 5.4 M actuellement). Même prévu de ferme en 2022
mais finalement annulé. Elle se troune aujourd'hui d'avantage vers la promotion des offres légales que dans la répression (sans pour autant l'abandonner).
\section{Fracture}
La fracture numérique est le nom donné à l'ensemble des disparités d'accès aux technologies informatiques, et notamment internet. 
Le droit d'accès à internet n'est pas encore reconnu mais de nombreuses discutions ont lieu en sa faveur (conseil constitutionnel et Hadopi). Une reconnaissance de ce droit amènerait l'obligation de réduire rapidement cette fracture numérique.
On la retrouve à plusieurs échelles, que ce soit au niveau mondial, national ou régional. La fracture numérique est principalement géographique, mais elle se retrouve également à un niveau social. Elle est fortement marquée entre les pays riches et les pays pauvres d'une part, et entre les métropoles et les zones de moyenne densité ou rurale d'autre part. 
On va principalement parler de la fracture géographique, tout d'abord à une échelle mondiale, et puis à l'échelle de la France. A chaque fois nous parlerons des projets en cours qui ont pour but de réduire cette fracture.
\subsection{Dans le monde}
photo:(2012)

ex : isalnde : 96 \% d'accès à internet, 1\% au Niger ou en Guinée
explication : principalement un manque de moyens financiers et de développement technologiques. Difficile de couvrir certaines zones (désert, montagnes). D'autres priorités. (je parle même pas du débit)

Solutions : 
\begin{itemize}
    \item Projet Loon by Google
    \url{https://www.qwant.com/?q=project%20loon&t=images&o=0:65b9113b85f748049e98ed1d17a10c3e}
    \url{https://s2.qwant.com/thumbr/0x0/7/d/77773190599e8b5653f477c5e6dd82/b_1_q_0_p_0.jpg?u=http%3A%2F%2Fimages.hngn.com%2Fdata%2Fimages%2Ffull%2F93252%2Fproject-loon.jpg&q=0&b=1&p=0&a=1}
    
    Ballons de 15m de diamètre gonflés à l'hélium, volant à à peu près 20km d'altitude, offrant un débit de connexion similaire à celui fournit par la 4G. 187j d'autonomie grâce à des panneaux solaires. transmet le réseau cellulaire.
    Déjà des expérimentations en Indonésie, Nouvelle-Zélande et Canada. Résultats plutôt satisfaisant même si certrains ballons se crashent. Accord passé avec le gouvernement Indonéisne et néo-zélandais pour donner accès ç internet à certaines zones. Nécessite l'isntallation d'antenne relais sur le territoire
    
    \item Projet Aquilla de Facebook
    \url{http://www.futura-sciences.com/tech/actualites/drone-aquila-drone-solaire-facebook-reussit-son-premier-vol-63674/} 
    \url{https://s2.qwant.com/thumbr/0x0/4/9/04e7f5cc9424bfdf6a2ab122628fe0/b_1_q_0_p_0.jpg?u=http%3A%2F%2Fwww.bionic3d.com%2Fstyle%2Fimages%2Fportfolio%2Faquila-drone-flying-illustration.jpg&q=0&b=1&p=0&a=1}
    Similaire à Google avec des drones ressemblant à des grands avions.  42m d'envergure . 90j d'autonomie à 17-26km d'altitude. Interent dans un rayon de 100km.
    Un seul essai, coûteux et nécessite encore des recherches (laser...).
    
\end{itemize}
\subsection{En France}
\begin{itemize}
    
\item
déploiement des réseaux mobiles (carte réseau 4G)
On parle de 4G, 4G+, 4G++ voir même de 5G alors que certaines villes n'ont aps encore accès à la 2G. On appelle zones blanches les zones sans accès à la 2G. Elle représente environ 3500 communes, qui représente ~1\% de la population française. 
En France, un organisme est chargé de gérer le déploiment des réseaux mobiles : l'ARCEP (autorité de régulation des communications électroniques et des postes).
Les objectifs de déploiement : pourcentage de la population, pourcentage du territoire, en 2G/3G/4G, axes routiers et ferroviaires, zones blanches
\\
Principaux objectifs fixés : 100\% des zones blanches en 2G fin 2016 (manqué) ,en 3G mi-2017, en 4G en 2027.
Zone peu dense (~63\% du terrtoire, 18\% de la population) en 4G en 2022
60\% du réseau ferroviaire 20 4G en 2022
https://www.monreseaumobile.fr/

\item
Déploiement de la fibre (carte observatoire très haut débit)
26\% de la pop qui à accès à la fibre avec des débits moyens largement supérieurs à 100Mbit/s alors que le débit moyen est à 10Mbit/s en France (classée 52ème mondiale, derrière la plupart des pays européens) et 7Mbit/s si on exclu ceux qui ont la fibre.

En 2013 est voté le plan très haut débit: il garantie à l'horizon 2022 100\% de la population couverte en très haut débit. (différent de la fibre, THD = >30Mbits donc d'autres technologies tq VDSL2, ou Coaxial)
\end{itemize}
\section{Conclusion}
La protection de la vie privée  en informatique est un droit qui permet de gérer le type d'informations collectées et par qui, de savoir ce qu'on collecte de nous et ce qu'on en fait.
La protection de la propriété intellectuelle est un droit pour protéger l'auteur d'un plagiat ou d'une contrefaçon.
L'Accessibilité numérique contre la fracture numérique est appuyée par le droit d'accès à internet pour réduire la fracture numérique. \\
Ouverture: Le droit numérique va poser des questions juridique quant à la frontière entre le respect de la vie privée et l'évolution dans une société de plus en plus en phase avec Internet et le partage en masse d'informations personnelles. Jusqu'à quel point pouvons nous accepter que le numérique empiète sur notre vie privée?
\end{document}