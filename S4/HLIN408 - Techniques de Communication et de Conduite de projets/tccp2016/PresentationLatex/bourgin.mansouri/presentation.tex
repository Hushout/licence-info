% Copyright 2004 by Till Tantau <tantau@users.sourceforge.net>.
%
% In principle, this file can be redistributed and/or modified under
% the terms of the GNU Public License, version 2.
%
% However, this file is supposed to be a template to be modified
% for your own needs. For this reason, if you use this file as a
% template and not specifically distribute it as part of a another
% package/program, I grant the extra permission to freely copy and
% modify this file as you see fit and even to delete this copyright
% notice. 

\documentclass{beamer}

% There are many different themes available for Beamer. A comprehensive
% list with examples is given here:
% http://deic.uab.es/~iblanes/beamer_gallery/index_by_theme.html
% You can uncomment the themes below if you would like to use a different
% one:
%\usetheme{AnnArbor}
%\usetheme{Antibes}
%\usetheme{Bergen}
%\usetheme{Berkeley}
%\usetheme{Berlin}
%\usetheme{Boadilla}
%\usetheme{boxes}
%\usetheme{CambridgeUS}
%\usetheme{Copenhagen}
%\usetheme{Darmstadt}
%\usetheme{default}
%\usetheme{Frankfurt}
%\usetheme{Goettingen}
%\usetheme{Hannover}
%\usetheme{Ilmenau}
%\usetheme{JuanLesPins}
%\usetheme{Luebeck}
%\usetheme{Madrid}
%\usetheme{Malmoe}
%\usetheme{Marburg}
\usetheme{Montpellier}
%\usetheme{PaloAlto}
%\usetheme{Pittsburgh}
%\usetheme{Rochester}
%\usetheme{Singapore}
%\usetheme{Szeged}
%\usetheme{Warsaw}

\title{Makefile et Ant}

\author{BOURGIN Jeremy\inst{1} \and MANSOURI Hind\inst{2}}
% - Give the names in the same order as the appear in the paper.
% - Use the \inst{?} command only if the authors have different
%   affiliation.

\institute[Université de Montpellier] % (optional, but mostly needed)
{
  \inst{1}%
  L2 INFORMATIQUE
  \and
  \inst{2}%
   Techniques de Communication et de Conduite de projets
}
% - Use the \inst command only if there are several affiliations.
% - Keep it simple, no one is interested in your street address.

\date{24/04/2017}
% - Either use conference name or its abbreviation.
% - Not really informative to the audience, more for people (including
%   yourself) who are reading the slides online

\subject{Theoretical Computer Science}
% This is only inserted into the PDF information catalog. Can be left
% out. 

% If you have a file called "university-logo-filename.xxx", where xxx
% is a graphic format that can be processed by latex or pdflatex,
% resp., then you can add a logo as follows:

% \pgfdeclareimage[height=0.5cm]{university-logo}{university-logo-filename}
% \logo{\pgfuseimage{university-logo}}

% Delete this, if you do not want the table of contents to pop up at
% the beginning of each subsection:
\AtBeginSubsection[]
{
  \begin{frame}<beamer>{plan}
    \tableofcontents[currentsection,currentsubsection]
  \end{frame}
}

% Let's get started
\begin{document}

\begin{frame}{}
    \titlepage
\end{frame}

\begin{frame}{Plan}
  \tableofcontents
  % You might wish to add the option [pausesections]
\end{frame}

% Section and subsections will appear in the presentation overview
% and table of contents.
\section{Makefile}

\subsection{Introduction}

\begin{frame}{GNU Make }
  \begin{itemize}
  \item {
   GNU Make est un outil qui contr\^ole la g\'en\'eration d'ex\'ecutables et d'autres fichiers non-source d'un programme \`a partir des fichiers source du programme.
  }
  \item {
    Make obtient sa connaissance de la facon de construire notre programme \`a partir d'un fichier appel\'e Makefile.
  }
  \end{itemize}
\end{frame}

\subsection{Compilation}

% You can reveal the parts of a slide one at a time
% with the \pause command:
\begin{frame}{Compilation d'un fichier et \'edition de liens}

    \begin{itemize}
        \item syntaxe des compilateurs
        \begin{itemize}
            \item compilation basique : 
                \begin{itemize}
                    \item g++  -o test fichier1.cpp fichier2.cpp 
                \end{itemize}
            \item l'\'edition de liens:
                \begin{itemize}
                    \item g++ -o fichier1.o -c fichier1.cpp
                    \item g++ -o fichier2.o -c fichier2.cpp
                    \item g++ fichier1.o  fichier2.o -o test
                \end{itemize}
        \end{itemize}
    \end{itemize}
\end{frame}

\subsection{Construire un projet avec make}

\begin{frame}{Compiler avec make en \'edition de lien}
   \begin{itemize}
        \item Syntaxe de make  
    \end{itemize}

    fichier1.o: fichier1.c
	\newline TAB g++ -c fichier1.c
    \newline
	\newline fichier2.o: fichier2.c
	\newline TAB g++ -c fichier2.c
    \newline
    \newline all : fichier1.o fichier2.o 
    \newline TAB g++ -o test fichier1.o fichier2.o
\end{frame}

\begin{frame}{ D\'efinition de variables}
  
  \begin{itemize}
    \item Variables personnalis\'ees
        \begin{itemize}
            \item CC
            \item CFLAGS
            \item LDFLAGS
            \item EXEC
        \end{itemize}
    
        \item   Variables internes
        
        \begin{itemize}
            \item \$@ : Le nom de la cible
            \item \$< : Le nom de la premi\`ere d\'ependance
            \item \$\^ : La liste des d\'ependances
            \item \$* : Le nom du fichier sans suffixe
        \end{itemize}
     \end{itemize}
\end{frame}


\begin{frame}
    Les r\'egles d'inf\'erence
    \begin{itemize}
        \item{Makefile permet \'egalement de cr\'eer des r\`egles g\'en\'eriques (par exemple construire un .o à partir d'un .cpp) qui se verront appel\'ees par d\'efaut.
        Une telle r\`egle se pr\'esente sous la forme suivante :}
    \end{itemize}
    \%.o: \%.cpp
    \newline TAB commandes
    \newline
    \begin{itemize}
        \item Construction de la liste des fichiers sources:
            \begin{itemize}
               \item SRC= \$(wildcard *.cpp)
    	   \end{itemize}
        \item G\'en\'eration de la liste des fichiers objets
            \begin{itemize}
               \item  OBJ= \$(patsubst \%.cpp,\%.o,\$(SRC))
    	   \end{itemize}
    \end{itemize}
\end{frame}

\begin{frame}{Utiliser make}
    Une fois le fichier makefile c\'e\'e, il suffit simplement de faire :\\
    make cible 
\end{frame}

\section{Ant}

\subsection{Introduction}

\begin{frame}{Introduction}
    \begin{block}{Cr\'e\'e par Apache}
        Ant a \'et\'e cr\'e\'e par la fondation Apache en 2000
    \end{block}
    \begin{block}{D\'evelopp\'e en Java}
        Ant est multiplate-forme
    \end{block}
    \begin{block}{Utilisation du XML}
        Standardis\'e, et simple de lecture
    \end{block}
    \begin{block}{Utilis\'e pour le Java}
        Ant a principalement \'et\'e cr\'e\'e pour automatiser les t\^aches en Java
    \end{block}
    \begin{block}{Des IDE utilisent Ant}
        Netbeans et Eclipse utilisent Ant pour construire des projets
    \end{block}
\end{frame}

\subsection{Java}

\begin{frame}{Java}
    \begin{itemize}
        \item cr\'e\'e en 1990
        \item Java est multiplate-forme (ind\'ependant de la plate-forme hardware)
        \item Orient\'e objet
        \item Pour utiliser Java, il faut installer le client JRE
        \item Pour d\'evelopper en Java :
    \end{itemize}
\end{frame}

\begin{frame}{D\'evelopper en Java}
    \begin{itemize}
        \item Installer Java JRE et JDK
        \item Configurer la variable PATH
        \item Configurer la variable JAVA\_HOME
        \item Un fichier pour une classe
        \item L'espace de nom (package) doit respecter l'emplacement du fichier dans l'arborescence du projet
        \item Pour compiler utiliser les commandes javac (g\'en\`ere un fichier binaire) et jar (g\'en\`ere un ex\'ecutable)
    \end{itemize}
\end{frame}

\subsection{Construire un projet avec Ant}

\begin{frame}{fichier de construction}
    Pour construire un projet, Ant attends le fichier build.xml. Ce fichier doit commencer par :\\
    $<$?xml version="1.0" encoding="UTF-8" ?$>$
    \newline
    \newline
    Et doit comporter la balise $<$project$>$ qui contiendra toute les instructions \`a la construction du projet 
\end{frame}

\begin{frame}{La balise project}
    $<$project name="hello" default="compile" dir="."$>$\\
    ...\\
    $<$/project$>$
    \newline
    \newline
    \begin{itemize}
        \item L'attribut name permet simplement de donner un nom à un projet
        \item L'attribut default permet de d\'esigner la cible par d\'efaut
        \item L'attribut dir permet d'indiquer l'emplacement du projet à construire
    \end{itemize}
\end{frame}

\begin{frame}{La balise property}
    La balise property permet de pouvoir cr\'eer des "variables" :\\
    $<$property name="project.sources.dir" value="src" /$>$\\
    $<$echo$>$\$\{project.sources.dir\}$<$/echo$>$
    \newline
    \newline
    \begin{itemize}
        \item Attribuer un nom \`a une variable avec l'attribut name
        \item Attribuer une valeur \`a une variable avec l'attribut value
        \item Pour utiliser une variable, il faut utiliser la syntaxe suivante : \$\{name\}
    \end{itemize}
\end{frame}

\begin{frame}{La balise path et pathelement}
    La balise path va permettre de regrouper un ensemble de dossiers et fichiers :\\
    $<$path id="classpath"$>$\\
    $<$pathelement location="projet.jar" /$>$\\
    $<$pathelement path="lib" /$>$\\
    $<$/path$>$
    \newline
    \newline
    \begin{itemize}
        \item L'attribut id va permettre de r\'ef\'erencer cet ensemble de dossiers et fichiers
        \item L'attribut location permet de d\'esigner un fichier
        \item L'attribut path permet de d\'esigner un dossier
    \end{itemize}
\end{frame}

\begin{frame}{La balise target}
    La balise target permet de cr\'eer des cibles :\\
    $<$target name="init" depends="clean"$>$\\
    $<$mkdir dir="bin" /$>$\\
    $<$/target$>$
    \newline
    \newline
    \begin{itemize}
        \item Attribuer un nom \`a une cible avec l'attribut name
        \item L'attribut depends permet d'appeler une autre cible
        \item Le corp comporte ce qui sera ex\'ecut\'e
    \end{itemize}
\end{frame}

\begin{frame}{Compiler avec la balise javac}
    $<$javac srcdir="src" destdir="bin" /$>$
    \newline
    \newline
    \begin{itemize}
        \item L'attribut srcdir d\'esigne le r\'epertoire o\`u se situe les fichiers sources
        \item L'attribut destdir d\'esigne le r\'epertoire o\`u seront envoy\'e les fichiers compil\'e
    \end{itemize}
\end{frame}

\begin{frame}{G\'en\'erer un ex\'ecutable avec la balise jar}
    $<$jar destfile="projet.jar" basedir="bin" /$>$
    \newline
    \newline
    \begin{itemize}
        \item L'attribut destfile d\'esigne le fichier ex\'ecutable qui sera cr\'e\'e
        \item L'attribut basedir d\'esigne le r\'epertoire o\`u sont stock\'e les fichiers compil\'e
    \end{itemize}
\end{frame}

\begin{frame}{Aller plus loin}
    \begin{itemize}
        \item Cr\'eer la documentation avec la balise $<$javadoc$>$
        \item Cr\'eer vos tests unitaires avec la balise $<$junit$>$
        \item Cr\'eer un seul fichier g\'en\'erique pour un ensemble de projet avec la balise $<$import$>$
        \item Vous pouvez aussi approfondir, et aller beaucoup plus loin avec Ant : http://ant.apache.org/
    \end{itemize}
\end{frame}

\begin{frame}{Utiliser Ant}
    Une fois le fichier build.xml c\'e\'e, il suffit simplement de faire :\\
    ant cible
\end{frame}

\section{D\'emonstration}

\section{Conclusion}

\begin{frame}{Conclusion}
  \begin{itemize}
  \item
    Utiliser make pour le c++ et le c
  \item
    Utiliser Ant pour le Java
  \item
    Le chef de projet doit g\'erer l'environnement de travail et les r\`egles de compilations
  \end{itemize}
\end{frame}

\end{document}
