\documentclass{beamer}
\usepackage[utf8]{inputenc}
\usepackage[T1]{fontenc}
\usepackage[francais]{babel}
\usepackage{tikz}
\usepackage{tikz-uml}
\usetikzlibrary{positioning}
\usetheme{Madrid}

\title{Introduction à l'UML}
\subtitle{Unified Modeling Language}
\author{W.Soussi \and B.Rima}
\institute[UM]{L2 CMI Informatique}
\date{HLIN408, \today{}}

\begin{document}

%titre
\begin{frame}
  \titlepage
\end{frame}

%sommaire
\begin{frame}{Sommaire}
  \tableofcontents
\end{frame}

%Section "Modèles et Langages de Modélisation"
\section{Modèles et Langages de Modélisation}

  %%Sous-section "Un Modèle"
  \subsection{Un Modèle}
    \begin{frame}{Un Modèle}{Modèles et Langages de Modélisation}
        \begin{Definition}
        Représentation abstraite et simplifiée d'une réalité.
        \pause
        \end{Definition}
        \begin{block}{Utilité}
          Idéalisation d'une réalité complexe et accentuation des détails importants.\\
          Description d'un problème et essai d'en trouver une solution.
        \end{block}
    \end{frame}

  %Sous-section "un langage de Modélisation"
  \subsection{Un Langage de Modélisation}
    \begin{frame}{Un Langage de Modélisation}{Modèles et Langages de Modélisation}
      \begin{Definition}
        Système formel permettant la modélisation d'une réalité.
       \pause
      \end{Definition}
      \begin{block}{Utilité}
        Définir les entités (caractéristiques, comportements, associations) composant un système complexe.
        \pause
      \end{block}
      \begin{block}{Remarque}
        Différent d'un langage de programmation, avec lequel il permet la mise en oeuvre d'un logiciel.
      \end{block}
    \end{frame}

%Section "L'UML"
\section{L'UML}

  %Sous-Section "Qu'est-ce que c'est l'UML"
  \subsection{Qu'est-ce que c'est l'UML}
    \begin{frame}{Qu'est-ce que c'est l'UML}{L'UML}
      \begin{itemize}
        \item{
          \textbf{UML} (\textbf{U}nified \textbf{M}odeling \textbf{L}anguage) est un langage de modélisation semi-formel et graphique utilisé pour:
            \begin{enumerate}
                \item {spécifier}
                \item {visualiser}
                \item {documenter}
            \end{enumerate}
          les artefacts d'un systéme.
          \pause
        }
        \item {Un \og artefact \fg{} est n'importe quel produit résultant de la réalisation d'un projet : fichiers sources, exécutables, documentations, fichiers de configurations etc.}
      \end{itemize}
    \end{frame}

    %2ème trame de la sous-section "Qu'est-ce que c'est l'UML"
    \begin{frame}{Qu'est-ce que c'est l'UML (2)}{L'UML}
      \begin{itemize}
        \item {Indépendant de la portée du projet.\pause}
        \item<2->{Indépendant du langage de programmation utilisé.}
        \item<3->{Composant d'un processus de développement et non pas le processus lui-même.}
      \end{itemize}
    \end{frame}

  %Sous-Section "L pour Langage"
  \subsection{L pour Langage}
    \begin{frame}{Un langage semi-formel}{L'UML}
      \begin{itemize}
       \item{Le langage UML est dit graphique parce qu'il est basé sur des diagrammes.\pause}
       \item<2->{Le langage UML est dit semi-formel parce qu'il est présenté par des diagrammes et par le langage naturel.}
       \item<3->{Un diagramme d'UML est constitué d'un ensemble d'éléments ayant une représentation graphique.}
       \item<4->{Ce n'est pas une notation graphique mais un vrai langage ayant des règles de syntaxe et de sémantique bien définies.}
      \end{itemize}
    \end{frame}

    %2ème trame de la sous-section "Un langage semi-formel"
    \begin{frame}{Un langage semi-formel (2)}{L'UML}
      \begin{itemize}
        \begin{Example}
           \begin{tikzpicture}
            \umlactor{Enseignant}
            \umlusecase[right=3cm of Enseignant]{enseigne}
            \umlactor[right=3cm of usecase-1]{Etudiant}
            \umluniassoc{Enseignant}{usecase-1}
            \umlassoc{usecase-1}{Etudiant}
          \end{tikzpicture}
        \end{Example}
        \item{SYNTAXE: La flèche indiquant le sens de l'association est facultative.}
        \item{SÉMANTIQUE: La flèche spécifie le sens de l'interprétation de l'association.}
      \end{itemize}
    \end{frame}

  %Sous-Section "un peu d'histoire"
  \subsection{Un peu d'histoire}
    \begin{frame}{Un peu d'histoire}{L'UML}
      \begin{itemize}
        \item {Fin des années 80 : plusieurs langages de modélisation objet --> un obstacle pour la diffusion du paradigme objet.\pause}
        \item<2->{1994 : RSC (Rational Software Corporation) fusionna les langages OOD (Booch) et OMT (Rumbaugh).}
        \item<3->{1995 : acquisition de l'entreprise d'Ivar Jacobson par RSC --> intégration de son langage OOSE aux deux autres (OOD et OMT).}
        \item<4->{1996 : Booch, Rumbaugh et Jacobson : chargés de la création de l'UML par RSC.}
      \end{itemize}
    \end{frame}

    %2ème trame de la sous-section "Un peu d'histoire"
    \begin{frame}{Un peu d'histoire (2)}{L'UML}
      \begin{itemize}
        \item {1997 : 1\up{ère} normalisation de UML par l'OMG (Object Management Group) : UML 1.0\pause}
        \item<2->{les versions d'UML les plus importantes adoptées par OMG :
          \begin{enumerate}
            \item {2005 : UML 2.0}
            \item {2007 : UML 2.1.2}
            \item {2013 : UML 2.5 beta 2}
          \end{enumerate}
        }
      \end{itemize}
    \end{frame}

    %Sous-Section "Le MOF"
    \subsection{Le MOF (Meta Object Facility)}
      \begin{frame}{Le MOF (Meta Object Facility)}{L'UML}
        \begin{itemize}
          \item {Architecture standardisée par OMG pour décrire formellement l'UML.\pause}
          \item<2->{Quatre niveaux de modélisation : M0, M1, M2 et M3.}
          \item<3->{Chaque niveau est une instance d'un élément du niveau supérieur :\pause
            \begin{enumerate}
              \item {Un élément de M0 est la réalité à décrire.\pause}
              \item{Un élément de M1 est un modèle décrivant une réalité en M0.\pause}
              \item{Un élément de M2 est un méta-modèle (ici l' UML) décrivant un modèle en M1.\pause}
              \item{Un élément de M3 est un méta-méta-modèle (ici le MOF) décrivant un méta-modèle en M2.\pause}
            \end{enumerate}
            }
            \item<4->{Un élément de M3 est défini comme une instance de lui-même.}
        \end{itemize}
      \end{frame}

%Section "Modèles et diagrammes d'UML"
\section{Modèles et diagrammes d'UML}

  %Sous-section "Modèles d'UML"
  \subsection{Modèles d'UML}
    \begin{frame}{Modèles d'UML}{Modèles et diagrammes d'UML}
      \begin{itemize}
        \item{UML décrit un système informatique selon quatre points de vue correspondant chacun à un \og \textit{modèle} \fg{} : \pause}
        \begin{enumerate}
          \item{modèle structurel/statique : décrire les types d'objets et leurs relations.\pause}
          \item{modèle dynamique : stimuli des objets et leurs réponses.\pause}
          \item{modèle d'utilisation : fonctionnalités des objets.\pause}
          \item{modèle d'implémentation : les composants, fichiers base de données, projection sur le matériel, etc...\pause}
        \end{enumerate}
        \item<2->{chaque modèle est une représentation abstraite d'une réalité fournissant une image simplifiée du monde réel selon un point de vue.}
      \end{itemize}
    \end{frame}

  %Sous-section "Diagrammes d'UML"
  \subsection{Diagrammes d'UML}
    \begin{frame}{Diagrammes d'UML}{Modèles et diagrammes d'UML}
      \begin{itemize}
        \item{chaque modèle d'UML contient des \og \textit{diagrammes} \fg{}, décrivant chacun des aspects particuliers du modèle : \pause}
        \begin{enumerate}
          \item{modèle structurel/statique : diagrammes de classes, d'instances.\pause}
          \item{modèle dynamique : diagrammes de séquences, d'états, d'activités.\pause}
          \item{modèle d'utilisation : diagramme de cas d'utilisation.\pause}
          \item{modèle d'implémentation : diagrammes de composants.}
        \end{enumerate}
      \end{itemize}
    \end{frame}

  %Sous-section "Diagrammes du modèle structurel"
  \subsection{Diagrammes du modèle structurel}
    \begin{frame}{Diagrammes du modèle structurel}{Modèles et diagrammes d'UML}
      \begin{block}{Diagramme d'instances}
        \begin{itemize}
          \item{décrire les objets du domaine modélisé.}
          \item{décrire les liens entre les objets.}
        \end{itemize}
      \end{block}
      \pause
      \begin{block}{Diagramme de classes}
        Une abstraction des diagrammes d'instances: des classes en relation regroupant chacune des objets ayant des caractéristiques communes.
      \end{block}
    \end{frame}

    %2ème trame de la sous-section "Diagrammes du modèle structurel"
    \begin{frame}{Diagrammes du modèle structurel (2)}{Modèles et diagrammes d'UML}
      \begin{block}{Syntaxe des attributs en UML}
        \og [visibilité] [/] nom [:type] [[multiplicité]] [= valeurParDéfaut] [\{propriétés...\}] \fg{}
        \begin{description}
          \item[visibilité :]{\{+, --, \#, $\sim$\}}
          \item[/ :]{le fait qu'un attribut soit dérivé}
          \item[nom :]{le nom de l'attribut (partie \textbf{obligatoire} de la syntaxe)}
          \item[type :]{le domaine de valeurs de l'attribut}
          \item[multiplicité :]{le nombre de valeurs que peut prendre l'attribut}
          \item[valeurParDéfaut :]{la valeur que possède l'attribut à l'origine}
          \item[propriétés :]{des propriétés précisant le comportement de l'attribut, (\{\textit{constant}\} pour indiquer qu'il est "read-only" par exemple)}
        \end{description}
      \end{block}
    \end{frame}

    %3ème trame de la sous-section "Diagrammes du modèle structurel"
    \begin{frame}{Diagrammes du modèle structurel (3)}{Modèles et diagrammes d'UML}
      \begin{block}{Syntaxe des opérations en UML}
        \og [visibilité] nom [(liste-paramètres)] [: typeRetour] [\{propriétés\}] \fg{}
        \begin{description}
          \item[visibilité :]{\{+, --, \#, $\sim$\}}
          \item[nom :]{le nom de l'opération (partie \textbf{obligatoire} de la syntaxe)}
          \item[(liste-paramètres) :]{les paramètres que peut avoir une opération}
          \item[typeRetour :]{le type de retour de l'opération}
          \item[propriétés :]{des propriétés précisant le comportement de l'opération, (\{\textit{query}\} pour indiquer que l'opération ne modifie pas l'instance courante sur laquelle elle est appliquée par exemple)}
        \end{description}
      \end{block}
    \end{frame}

    %4ème trame de la sous-section "Diagrammes du modèle structurel"
    \begin{frame}{Diagrammes du modèle structurel (4)}{Modèles et diagrammes d'UML}
      \begin{block}{Syntaxe des paramètres des opérations en UML}
        \og [direction] nom : type [multiplicité] [= valeurParDéfaut] \{propriétés\} \fg{}
        \begin{description}
          \item[direction :]{\{in, out, inout\}}
          \item[nom :]{le nom du paramètre (partie \textbf{obligatoire} de la syntaxe)}
          \item[type :]{le domaine de valeurs du paramètre (partie \textbf{obligatoire} de la syntaxe)}
          \item[multiplicité :]{le nombre de valeurs que peut prendre le paramètre}
          \item[valeurParDéfaut :]{la valeur que possède le paramètre à l'origine}
          \item[propriétés :]{des propriétés précisant le comportement du paramètre, (\textit{idem} que les attributs)}
        \end{description}
      \end{block}
    \end{frame}

    %1ère trame d'exemples de la sous-section "Diagrammes du modèle structurel"
    \begin{frame}{Diagrammes du modèle structurel (Exemples)}{Modèles et diagrammes d'UML}
      \begin{figure}
        \begin{tikzpicture}
          \begin{umlpackage}{véhicules}
            \umlclass{Voiture}{
              -- marque : String \\
              -- type : String \\
              -- couleur : int \\
              \umlstatic{-- nbRoues : int = 4 \{cst\}}
            }
            {
              + klaxonner() \\
              \umlstatic{+ getNbRoues() : int \{query\}}
            }
            \umlclass[right=3cm of Voiture]{porsche:Voiture}{
              "Porsche" \\
              "Cayenne" \\
              "Jaune"
            }
            {}
            \umldep[stereo=instance]{porsche:Voiture}{Voiture}
          \end{umlpackage}
        \end{tikzpicture}
      \end{figure}
    \end{frame}

    %2ème trame d'exemples de la sous-section "Diagrammes du modèle structurel"
    \begin{frame}{Diagrammes du modèle structurel (Exemples)}{Modèles et diagrammes d'UML}
      \begin{figure}
        \begin{tikzpicture}
          \umlclass[type=abstract]{Véhicule}{
            -- nom : String \\
            -- couleur : String \\
            -- puissance : double
          }
          {
            \umlvirt{+ avancer(d : double)}
          }
          \umlclass[x=4, y=-4]{Voiture}{
            -- marque : String \\
            \umlstatic{-- nbRoues : int = 4 \{cst\}}
          }
          {
            + klaxonner() \\
            + avancer(d : double) \\
            + freiner() \\
            \umlstatic{+ getNbRoues() : int \{query\}}
          }
          \umlclass[x=-3, y=-4]{Bateau}{
            -- carburant : String \\
          }
          {
            + avancer(d : double)\\
            + chavirer()
          }
          \umlnote[x=5, width=2cm]{Véhicule}{Superclasse}
          \umlnote[y=-4, width=1cm]{Bateau}{Sous-classes}
          \umlnote[y=-4, width=1cm]{Voiture}{Sous-classes}
          \umlVHVinherit{Bateau}{Véhicule}
          \umlVHVinherit{Voiture}{Véhicule}
        \end{tikzpicture}
      \end{figure}
    \end{frame}

    %1ère trame d'associations de la sous-section "Diagrammes du modèle structurel"
    \begin{frame}{Diagrammes du modèle structurel (Les associations)}{Modèles et diagrammes d'UML}
      \begin{Definition}
        Une association est une abstraction de liens identifiés entre des instances de classes.
      \end{Definition}

      \begin{Example}
        \begin{tikzpicture}
          \umlsimpleclass{Personne}
          \umlsimpleclass[x=8]{Voiture}
          \umluniassoc[stereo=posséder, arg1=propriétaire, mult1=0..2, arg2=voiture, mult2=*]{Personne}{Voiture}
        \end{tikzpicture}
        \pause
      \end{Example}

      \begin{block}{Remarque}
        L'association est une notion de modélisation qui est absente en programmation.
      \end{block}
    \end{frame}

    %2ème trame d'associations de la sous-section "Diagrammes du modèle structurel"
    \begin{frame}{Diagrammes du modèle structurel (Les associations)}{Modèles et diagrammes d'UML}
      \begin{block}{Agrégation}
        L'agrégation est une association entre un ensemble d'éléments et un autre élément.
      \end{block}

      \begin{Example}
        \begin{tikzpicture}
          \umlsimpleclass{Personne}
          \umlsimpleclass[x=8]{Voiture}
          \umluniaggreg[stereo=posséder, arg1=propriétaire, mult1=0..2, arg2=voiture, mult2=*]{Personne}{Voiture}
        \end{tikzpicture}
      \end{Example}
    \end{frame}

  %3ème trame d'associations de la sous-section "Diagrammes du modèle structurel"
  \begin{frame}{Diagrammes du modèle structurel (Les associations)}{Modèles et diagrammes d'UML}
    \begin{block}{Composition}
      La composition est une agrégation où l'ensemble d'éléments, les composants, appartiennent à un seul élément, nommé composite.
    \end{block}

    \begin{Example}
      \begin{tikzpicture}
        \umlsimpleclass{CapitalSociale}
        \umlsimpleclass[x=9]{Action}
        \umlunicompo[stereo=composer, arg1=CapitalSociale, mult1=1, arg2=participation, mult2=*]{CapitalSociale}{Action}
      \end{tikzpicture}
    \end{Example}
  \end{frame}

  %4ème trame d'associations de la sous-section "Diagrammes du modèle structurel"
  \begin{frame}{Diagrammes du modèle structurel (Les associations)}{Modèles et diagrammes d'UML}
    \begin{block}{Classe d'association}
      Une classe d'association est une classe qui contient les attributs de lien d'une association.
    \end{block}

    \begin{Example}
      \begin{tikzpicture}
        \umlsimpleclass{Magasin}
        \umlsimpleclass[x=9]{Personne}
        \umlunicompo[stereo=acheter, name=compo, arg1=client, mult1=*, arg2=magasin, mult2=*]{Personne}{Magasin}
        \umlassocclass[x=4,y=-2.6]{Achat}{compo-1}{
          -- magasin : Magasin \\
          -- client : Personne \\
          -- date : Date \\
          -- produits : String[] \\
        }
        {}
      \end{tikzpicture}
    \end{Example}
  \end{frame}

%Section "Outils et logiciels pour UML"
\section{Outils et logiciels pour UML}

    %Trame d'exemple de la section "Outils et logiciels pour UML"
    \begin{frame}{Exemples}{Outils et logiciels pour UML}
      \begin{itemize}
        \item {Il existe plein de logiciels et de services en ligne pour :\pause}
        \begin{enumerate}
          \item {Dessiner en UML.\pause}
          \item {Générer d'UML à partir de code source.\pause}
          \item {Générer du code source à partir d'UML.\pause}
          \item {etc.}
        \end{enumerate}
        \item {Parmi les outils les plus couramment utilisés : }
        \begin{enumerate}
          \item {draw.io
            (\textcolor{blue}{
              \underline{
                \url{
                  https://www.draw.io/
                })\pause
              }
            }
          }
          \item {Umbrello
            (\textcolor{blue}{
              \underline{
                \url{
                  https://umbrello.kde.org/
                })\pause
              }
            }
          }
          \item {
            tikz-uml : un package en LaTeX pour faire des diagrammes d'UML
            (\textcolor{blue}{
              \underline{
                \url{
                  http://perso.ensta-paristech.fr/~kielbasi/tikzuml/
                })\pause
              }
            }
          }
          \item {ObjectAid : un plugin eclipse pour générer de l'UML
            (\textcolor{blue}{
              \underline{
                \url{
                  http://www.objectaid.com/home
                })
              }
            }
          }
        \end{enumerate}
      \end{itemize}
    \end{frame}

%Section "Conclusion" non incluse dans le sommaire
\section*{Conclusion}

  %Trame de la conclusion de la section "Conclusion"
  \begin{frame}{Conclusion}
    \begin{itemize}
      \item {\alert{Un Modèle}: Représentation abstraite et simplifiée d'une réalité.}
      \item {\alert{Un Langage de Modélisation}: Système formel permettant la modélisation d'une réalité.}
      \item {\alert{UML}: Langage de modélisation semi-formel graphique.}
      \item {\alert{Un Diagramme d'UML}: Une entité graphique en UML décrivant des aspects particuliers d'un modèle UML.}
      \item {Parmi les diagrammes d'UML : \alert{le diagramme de classes}, \alert{d'instances}, et le \alert{diagramme de cas d'utilisation}.}
      \item {Parmi les outils permettant de manipuler l'UML : \alert{ObjectAid} pour eclipse et \alert{tikz-uml} en LaTeX.}
    \end{itemize}
  \end{frame}

%Section "Bibliographié" non incluse dans le sommaire
\section*{Bibliographie}

  %Trame de "bibliographie" de la section "Bibliographie"
  \begin{frame}[allowframebreaks]

    \frametitle<presentation>{Bibliographie}
    \begin{thebibliography}{}
      \bibitem{HLIN406}
      Marianne Huchard, Clémentine Nebut.
      \newblock{Modélisation et programmation par Objets 1 HLIN406, 12 janvier 2017}

      \bibitem{Favini}
      Gian Piero Favini
    \newblock {\em{Introduzione à l'UML}.}
      www.cs.unibo.it/gabbri/MaterialeCorsi/1.introUML.favini.pdf, 2007.
    \end{thebibliography}

  \end{frame}
\end{document}

% DIAPORAMAS SUPPLÉMENTAIRES SUR UML NON INCLUS DANS LA PRÉSENTATION
%===================================================================
% \subsection{Concepts fondamentaux en l'approche par objet}
%   \begin{frame}{Concepts fondamentaux en l'approche par objet (1/8)}{Modèles et diagrammes d'UML}
%     \begin{block}{Objet (Instance)}
%     Une entité symbolisant un concept physique ou abstrait du monde réel, caractérisée par des données (\textit{attributs}) et des comportements (\textit{opérations/méthodes}).
%     \end{block}
%
%     \begin{Example}
%     Bachar Rima ayant 24 ans, capable de dire bonjour et de manger.\\
%     Wissem Soussi ayant 20 ans, capable de dire bonjour et de manger.\\
%     Une porsche cayenne jaune, capable de klaxonner.
%     \end{Example}
%   \end{frame}
%
%   \begin{frame}{Concepts fondamentaux en l'approche par objet (2/8)}{Modèles et diagrammes d'UML}
%     \begin{block}{Classe}
%     Un concept d'abstraction permettant la description formelle d'un ensemble d'objets ayant une sémantique et des caractéristiques communes.
%     \end{block}
%
%     \begin{Example}
%     Une classe Personne caractérisée par un nom, un prénom, un age, et les opérations bonjour() et manger().\\
%     Une classe Voiture caractérisée par un type, une marque, une couleur, et l'opération klaxonner().
%     \end{Example}
%   \end{frame}
%
%   \begin{frame}{Concepts fondamentaux en l'approche par objet (3/8)}{Modèles et diagrammes d'UML}
%     \begin{block}{Instanciation}
%     L'action permettant de créer d'objets à partir d'une classe en affectant des valeurs à leurs attributs.
%     \end{block}
%
%     \begin{block}{Constructeur}
%     Une opération particulière permettant d'instancier des objets à partir d'une classe.
%     \end{block}
%
%     \begin{block}{Destructeur}
%     Une opération particulière permettant de détruire des objets d'une classe.
%     \end{block}
%   \end{frame}
%
%   \begin{frame}{Concepts fondamentaux en l'approche par objet (4/8)}{Modèles et diagrammes d'UML}
%     \begin{block}{Attribut de classe}
%     Un attribut descriptif de la classe elle-même, plutôt que d'une instance de cette classe, et dont la valeur est partagée par toutes les instances de la classe.
%     \end{block}
%
%     \begin{block}{Opération de classe}
%     Une opération qui ne s'applique qu'à des attributs et des opérations de classe.
%     \end{block}
%
%     \begin{Example}
%     Toutes les voitures ont quatre roues, donc le nombre de roues d'une voiture est un attribut de classe de la classe Voiture.\\
%     L'opération getNbRoues permettant de récupérer le nombre de roues d'une voiture est une opération de classe.
%     \end{Example}
%   \end{frame}
%
%   \begin{frame}{Concepts fondamentaux en l'approche par objet (5/8)}{Modèles et diagrammes d'UML}
%     \begin{block}{Énumération}
%     Un type de données dont on peut énumérer toutes les valeurs possibles.
%     \end{block}
%
%     \begin{Example}
%     La civilité d'une personne peut être assimilée à une énumération dont les valeurs sont: M., Mme. et Mlle
%     \end{Example}
%   \end{frame}
%
%   \begin{frame}{Concepts fondamentaux en l'approche par objet (6/8)}{Modèles et diagrammes d'UML}
%     \begin{block}{Paquetage}
%     Un regroupement logique d'éléments UML (classes, énumérations, etc...) permettant de structurer l'application les utilisant et d'assurer une bonne traçabilité de l'analyse à l'implémentation.
%     \end{block}
%
%     \begin{Example}
%     Un paquetage étudiant contenant des classes Etudiant, Faculté, Promotion, etc...
%     \end{Example}
%   \end{frame}
%
%   \begin{frame}{Concepts fondamentaux en l'approche par objet (7/8)}{Modèles et diagrammes d'UML}
%     \begin{block}{Héritage}
%     Deux classes A et B réliées par une relation \og \textit{B est forcément A} \fg{} mais \og \textit{A n'est pas forcément B} \fg{} sont désignées par une relation d'héritage, telle que B hérite les caractéristiques \textit{(attributs/opérations)} de A.\\
%     Par conséquent, A est appelée \textit{superclasse} de B et B est appelée \textit{sous-classe} de A.
%     \end{block}
%
%     \begin{Example}
%     Soient deux classes Personne et Étudiant; on remarque bien qu'un étudiant est forcément une personne mais qu'une personne n'est pas forcément un étudiant. Ainsi on peut dire que la classe Étudiant hérite de la classe Personne.
%     \end{Example}
%   \end{frame}
