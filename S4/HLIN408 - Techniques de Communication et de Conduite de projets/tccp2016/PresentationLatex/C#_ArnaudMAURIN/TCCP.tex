\documentclass{beamer}
\usepackage[utf8]{inputenc}
 \usepackage[T1]{fontenc}

\usetheme{Madrid}

\title{Présentation du langage C\#}

\subtitle{Et de l'interface de développement Visual Studio}

\author{Arnaud MAURIN}

\institute[Université de Montpellier] 
{
  Licence 2 informatique\\
  HLIN408 - TCCP
}


\date{Semestre 4, 2017}

\subject{Présentation du langage C\#}

\begin{document}

\begin{frame}
  \titlepage
\end{frame}

\begin{frame}{Sommaire}
  \tableofcontents
\end{frame}

\section{Introduction}

    \begin{frame}{}
    \LARGE I. Introduction au langage et à son interface de développement
    \end{frame}

\subsection{Le Langage C\#}

\begin{frame}{Le langage C\# :}
  \begin{itemize}
  \item {
    Conçu en 2000 par Microsoft, finalisé en 2002
    \pause
  }
  \item {   
    Fonctionne avec le Framework .NET sur Windows, toujours par Microsoft
    \pause
  }
  \item {
    Fonctionne avec Mono ou DotGNU sur Linux et macOS
    \pause
  }
  \item {
    Permet du développement de clients lourds, de clients légers (via ASP.net), d'applications desktop ou mobiles (sur Windows Phone)
  }
  \end{itemize}
\end{frame}

\subsection{Caractéristiques du C\#}

\begin{frame}{Caractéristiques du C\# :}
  \begin{itemize}
  \item {
    Compilable, dérivé de C++, proche de Java en terme de syntaxe et de fonctionnement
    \pause
  }
  \item {   
    Gestion implicite des pointeurs
    \pause
  }
  \item {
    Gestion d'exceptions, ramasse-miettes
    \pause
  }
  \item {
    Tout objet, héritage (de classe unique seulement), polymorphisme..
  }
  \end{itemize}
\end{frame}

\subsection{Microsoft Visual Studio}

\begin{frame}{Microsoft Visual Studio :}
  \begin{itemize}
  \item {
    Compatible Windows et macOS prochainement (est remplacée par MonoDevelopp sur Linux et Xamarin sur Mac)
    \pause
  }
  \item {   
    Permet de développer en Visual Basic, Visual C++, Visual C\#, ASP.net
    \pause
  }
  \item {
    Peut développer les applications desktop comme mobile
    \pause
  }
  \item {
    Entièrement traduite en français, très automatisée.
  }
  \end{itemize}
\end{frame}

\section{Le développement de clients lourds}
    \begin{frame}{}
    \LARGE II. Le développement de clients lourds
    \end{frame}

\subsection{Edition en POO}

\begin{frame}{Edition en POO :}
  \begin{itemize}
  \item {
    Pas d'obligation d'organisation (une classe par fichier/plusieurs classes par fichiers)
    \pause
  }
  \item {   
    Possibilité de générer automatiquement constructeurs, accesseurs, modificateurs...
    \pause
  }
  \item {
    Possibilité d'éditer les classes directement via le diagrame UML
    \pause
  }
  \end{itemize}
\end{frame}

\subsection{Interfaces graphiques}

\begin{frame}{Interfaces graphiques :}
  \begin{itemize}
  \item {
    Interface graphique gérée par la librairie WindowsForms
    \pause
  }
  \item {   
    Conception entièrement automatisée se faisant via l'interface du logiciel
    \pause
  }
  \item {
    Gestion des évènements automatique également
    \pause
  }
  \end{itemize}
\end{frame}

\subsection{Bases de données lourdes}

\begin{frame}{Bases de données lourdes :}
  \begin{itemize}
  \item {
    Prise en charge de Microsoft SQL Server
    \pause
  }
  \item {   
    Lien via Entity Framework
    \pause
  }
  \item {
    Liaison à l’interface via l’objet Binding Source
    \pause
  }
  \end{itemize}
\end{frame}

\section{Le développement pour appareil mobile (Windows Phone)}
    \begin{frame}{}
    \LARGE II. Le développement pour appareil mobile (Windows Phone)
    \end{frame}

\subsection{Outils de développement nécessaires}

\begin{frame}{Outils de développement nécessaires}
  \begin{itemize}
  \item {
    Une machine fonctionnant sous Windows (avec Visual Studio)
    \pause
  }
  \item {   
    Un téléphone Windows Phone (ou l’émulateur)
    \pause
  }
  \item {
    Le Kit de développement
    \pause
  }
  \end{itemize}
\end{frame}

\subsection{Interfaces graphiques}

\begin{frame}{Test de l’application :}
  \begin{itemize}
  \item {
    Créer un compte Développeur Microsoft
    \pause
  }
  \item {   
    Acheter une licence de développement
    \pause
  }
  \item {
    Enregistrer le téléphone en tant qu’appareil développeur
    \pause
  }
  \end{itemize}
\end{frame}

\subsection{Base de données sur appareil mobile}

\begin{frame}{Base de données sur appareil mobile :}
  \begin{itemize}
  \item {
    Codées en LINQ, stockées à part
    \pause
  }
  \item {   
    Gestion intégrée via l’objet Datacontext du .NET
    \pause
  }
  \item {
    Déclaration via une surcharge des classes
    \pause
  }
  \end{itemize}
\end{frame}

\section{Démo}
    \begin{frame}{}
    \LARGE Time for a little Demo !!!
    \end{frame}

\end{document}