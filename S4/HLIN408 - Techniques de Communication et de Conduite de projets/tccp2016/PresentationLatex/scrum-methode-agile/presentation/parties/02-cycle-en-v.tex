\section{Une méthode classique : le cycle en V}

\begin{frame}{Point d’histoire}
    \begin{itemize}
        \item Mis au point par les départements de défense étasunien et allemand.
        \item Initialement destiné aux systèmes de satellites.
        \item Adapté à l’ingénierie logicielle dans les années 1980.
    \end{itemize}
\end{frame}

\begin{frame}{Comment fonctionne le cycle en V ?}
    \tikzset{
        phasebox/.style={
            rectangle,
            thick,
            align=center,
            draw=fg,
            minimum height=.75cm
        }
    }

    \begin{tikzpicture}
        {[
            node distance=.5cm,
            every node/.style={phasebox},
            edge from parent/.style={},
            level distance=1.6cm,
            level 1/.style={sibling distance=2.5cm},
            level 2/.style={sibling distance=1.5cm}
        ]
            % Arbre en V
            \node [visible on=<4->] (codage) {Programmation} [grow'=up]
            child {
                node [visible on=<3->] (conc-det) {Conception\\modulaire}
                child {
                    node [visible on=<2->](conc-arc) {Conception\\architecturale}
                    child {
                        node (anal-bes) {Analyse des\\besoins}
                        child[missing]
                    }
                    child[missing]
                }
                child[missing]
            }
            child {
                node [visible on=<5->] (test-unit) {Test\\unitaire}
                child[missing]
                child {
                    node [visible on=<6->] (test-int) {Test\\d’intégration}
                    child[missing]
                    child {
                        node [visible on=<7->] (test-acc) {Test\\d’acceptation}
                    }
                }
            };
        }

        % Flèches ordonnées
        \draw[->, arrow, visible on=<2->] (anal-bes) -- (conc-arc);
        \draw[->, arrow, visible on=<3->] (conc-arc) -- (conc-det);
        \draw[->, arrow, visible on=<4->] (conc-det) -- (codage);
        \draw[->, arrow, visible on=<5->] (codage) -- (test-unit);
        \draw[->, arrow, visible on=<6->] (test-unit) -- (test-int);
        \draw[->, arrow, visible on=<7->] (test-int) -- (test-acc);

        % Flèches de retour
        \draw[->, arrow, visible on=<5->] (test-unit) -- (conc-det);
        \draw[->, arrow, visible on=<6->] (test-int) -- (conc-arc);
        \draw[->, arrow, visible on=<7->] (test-acc) -- (anal-bes);

        % Rôles
        \node[image, right=of test-acc] (role-user)
            {\includegraphics[width=1cm]{icon-user}\\Utilisateur};

        \node[image, visible on=<4->] (role-coding) at
            (codage -| role-user)
            {\includegraphics[width=1cm]{icon-coding}\\Programmeurs};

        \node[image, visible on=<3->] (role-engineer) at
            ($(role-coding)!0.5!(role-user)$)
            {\includegraphics[width=1cm]{icon-engineer}\\Ingénieurs};

        \draw[
            decoration={brace, raise=1.75cm},
            thick, decorate, visible on=<8->
        ] (role-user.north) --
            node[image, right=2.25cm]
            {\includegraphics[width=1cm]{icon-boss}\\Chef de\\projet}
          (role-coding.south);
    \end{tikzpicture}
\end{frame}

\begin{frame}{Avantages et inconvénients}
    \begin{itemize}
        \itemcheck Simple à comprendre et à appréhender.
        \itemcheck Standard historique de l’industrie.
        \itemcheck Fixe un accord avec le client sur les spécifications.
        \itemcheck Intègre les tests au processus de création.
    \end{itemize}
    \pause
    \begin{itemize}
        \itemcross Crée une hiérarchie importante.
        \itemcross Déconnexion entre le client et la création du produit.
        \itemcross Désynchronisation entre conception et implantation.
        \itemcross Pas suffisamment flexible (changement de besoin du client).
    \end{itemize}
\end{frame}
