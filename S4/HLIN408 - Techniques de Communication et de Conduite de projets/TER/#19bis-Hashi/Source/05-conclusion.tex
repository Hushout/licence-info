\chapter*{Conclusion et remerciements}
\addcontentsline{toc}{chapter}{Conclusion et remerciements}
\markboth{Conclusion}{Conclusion}
\label{chap:conclusion}

Ce projet nous a tout d'abord permis d'appliquer nos connaissances sur un problème concret et ainsi d' enrichir notre expérience. En effet, le fait de réaliser un projet de cette envergure nous a servi à nous tester sur nos compétences en programmation. Nous avions eu, d'ores et déjà au premier semestre, un défi de ce genre avec le jeu TenTen. Seulement, celui-ci, ne proposait pas la réflexion que nous a posé le Hashi puisqu'il s'agissait simplement d'un jeu et non d'un casse-tête. C'est pourquoi, avoir eu l'opportunité de travailler sur des projets comme celui-ci a pu améliorer notre expérience général, que ce soit en terme de programmation mais aussi de réflexion.\newline
Il nous a aussi entraîné à améliorer la gestion de notre temps, qui n'a pas toujours été optimale. Effectivement, afin de produire dans les temps un projet, il nous faut de l'organisation. Ce type "d'épreuve" permet ainsi de se prendre en main en nous faisant réaliser l'ampleur de ce qu'il y a faire. Cela nous aide à être encore plus autonome dans notre travail mais, pour autant, cela nous aide aussi à apprendre à travailler en groupe. Ces dernières notions sont d'autant plus importantes lorsqu'il s'agira de réaliser la demande de notre employeur ou de notre client quand nous trouverons un boulot après nos études. Ainsi, nous pouvons dire que ce type de projet est très important pour nous afin d'être au plus prêt lorsque cela arrivera.
\smallbreak
Malgré tout, notre programme n'est pas complet et ne permet pas de résoudre les problèmes les plus complexes, qui sont les cas où plus aucune de nos règles ne peuvent être appliquées. En effet, dans certaines configurations de départ, il se peut qu'aucune règle ne puisse être mis en oeuvre. Dans ce cas là, il faut alors commencer par essayer différentes combinaisons de ponts entre les îles de manière aléatoire et ainsi débloquer la situation, entraînant l'activation d'une des règles. Néanmoins, il est possible que la façon de commencer ne soit pas la bonne et mène à nulle part. Il faut alors repartir de zéro et, à nouveau, recommencer d'une autre manière jusqu'à trouver l'unique et bonne solution.\newline
Nous n'avons pas aussi eu le temps de s'attarder sur les règles concernant la connexité. Effectivement, il en existe et servent éventuellement à débloquer certaines situations ou du moins à les compléter. Dans notre version du Hashi, nous nous sommes penchés sur la connexité finale, c'est-à-dire, si nous avons bien qu'une seule composante connexe à la fin mais nous n'avons créé aucune règle similaire à celles existantes les concernant. Or il s'avère qu'elles peuvent permettre la création de ponts. Malgré cela, ces dernières ne sont pas indispensables, du moins, elles n'apparaissent que dans de rares cas.
\smallbreak
Pour finir, nous tenons particulièrement à remercier notre encadrant M. Philippe Janssen qui a su nous aider tout au long du projet grâce aux réunions hebdomadaires et notamment à l'aide de ses remarques sur notre travail mais aussi à l'aide de ses remarques sur notre organisation de travail. Cette aide nous a permis de nous débloquer à maintes reprises et nous a permis de savoir quoi faire lors de certaines situations.\newline
