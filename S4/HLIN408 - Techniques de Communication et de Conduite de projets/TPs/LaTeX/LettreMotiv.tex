%% start of file `template.tex'.
%% Copyright 2006-2013 Xavier Danaux (xdanaux@gmail.com).
%
% This work may be distributed and/or modified under the
% conditions of the LaTeX Project Public License version 1.3c,
% available at http://www.latex-project.org/lppl/.


\documentclass[11pt,a4paper,sans]{moderncv}        % possible options include font size ('10pt', '11pt' and '12pt'), paper size ('a4paper', 'letterpaper', 'a5paper', 'legalpaper', 'executivepaper' and 'landscape') and font family ('sans' and 'roman')

% moderncv themes
\moderncvstyle{casual}                             % style options are 'casual' (default), 'classic', 'oldstyle' and 'banking'
\moderncvcolor{blue}                               % color options 'blue' (default), 'orange', 'green', 'red', 'purple', 'grey' and 'black'
%\renewcommand{\familydefault}{\sfdefault}         % to set the default font; use '\sfdefault' for the default sans serif font, '\rmdefault' for the default roman one, or any tex font name
%\nopagenumbers{}                                  % uncomment to suppress automatic page numbering for CVs longer than one page
\usepackage[french]{babel}
% character encoding
\usepackage[utf8]{inputenc}                       % if you are not using xelatex ou lualatex, replace by the encoding you are using
%\usepackage{CJKutf8}                              % if you need to use CJK to typeset your resume in Chinese, Japanese or Korean

% adjust the page margins
\usepackage[scale=0.75]{geometry}
%\setlength{\hintscolumnwidth}{3cm}                % if you want to change the width of the column with the dates
%\setlength{\makecvtitlenamewidth}{10cm}           % for the 'classic' style, if you want to force the width allocated to your name and avoid line breaks. be careful though, the length is normally calculated to avoid any overlap with your personal info; use this at your own typographical risks...

% personal data
\name{Thibault}{Odorico}
\title{Resumé title}                               % optional, remove / comment the line if not wanted
\address{215 Chemin de la filature}{30500 Saint-Ambroix}{France}% optional, remove / comment the line if not wanted; the "postcode city" and and "country" arguments can be omitted or provided empty
\phone[mobile]{06~21~79~68~11}                   % optional, remove / comment the line if not wanted
\phone[fixed]{04~66~24~02~61}                    % optional, remove / comment the line if not wanted
\email{thibault.odorico@etu.umontpellier.fr}                               % optional, remove / comment the line if not wanted
\photo[64pt][0.4pt]{picture}                       % optional, remove / comment the line if not wanted; '64pt' is the height the picture must be resized to, 0.4pt is the thickness of the frame around it (put it to 0pt for no frame) and 'picture' is the name of the picture file
\quote{Some quote}                                 % optional, remove / comment the line if not wanted

% to show numerical labels in the bibliography (default is to show no labels); only useful if you make citations in your resume
%\makeatletter
%\renewcommand*{\bibliographyitemlabel}{\@biblabel{\arabic{enumiv}}}
%\makeatother
%\renewcommand*{\bibliographyitemlabel}{[\arabic{enumiv}]}% CONSIDER REPLACING THE ABOVE BY THIS

% bibliography with mutiple entries
%\usepackage{multibib}
%\newcites{book,misc}{{Books},{Others}}
%----------------------------------------------------------------------------------
%            content
%----------------------------------------------------------------------------------
\begin{document}
%-----       letter       ---------------------------------------------------------
% recipient data
\recipient{Master Imagina}{Faculté des sciences et des lettres\\Place Eugène Bataillon\\34090 Montpellier}
\date{\today}
\opening{Cher Monsieur/Chère Madame,}
\closing{Je vous prie, Monsieur/Madame, d’agréer l’expression de mes salutations distinguées,}
\makelettertitle


Actuellement étudiant en troisième année de licence d’informatique à l’Université de
Montpellier, je souhaiterais m’orienter vers le Master IMAGINA que vous proposez.\\

Cette voie est pour moi celle qui me convient le mieux en effet, ayant 
toujours etait passionné par le traitement de l'image et la programmation de jeux vidéos, 
j’aimerais etudier ces domaines qui me plaisent tant et vivre de cette passion.\\

De plus, ayant déjà profité de mon cursus en licence pour approfondir mes
connaissances en informatique en général mais egalement dans la programmation de jeux video avec notamment 
la création d'un Pacman sur le moteur de jeux Unity, ce Master s’inscrirait parfaitement dans la logique de mon
projet professionnel.\\

Je dispose de la capacité de travail et de la motivation nécessaires à la réussite d’un
Master, et je sollicite donc votre bienveillance à l’égard de mon dossier pour mon admission à
ce Master.\\

Je vous remercie par avance de l’attention que vous porterez à ma demande, et reste à votre
entière disposition.\\

\vspace{1\baselineskip}

\makeletterclosing

\end{document}


%% end of file `template.tex'.
